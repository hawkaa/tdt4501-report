\chapter{Background Information}
\label{chap:background}
\begin{secex}
This chapter will introduce relevant background information.
\begin{enumerate}
  \item Analytical vs Tranasctional workloads. How they normally was a separate system, but wants to unite. Use the Pavlic thesis on this. Star and snowflake schema?
  \item Background on column stores.
  \item Background on in-memory storages.
  \item Background on compression.
  \item Background on business intelligence and how it evolved into business discovery. Should explain the nature of the queries, and that they cannot be anticipated. Perhaps something about non-SQL type of queries.
  \item Predicate evaluation
  \item Background on testing (TPC-H, Snowflake, TPC-CH ++)
  \item Background on which vendors are out there. Explain the TPC-H benchmark, and which vendors are on top.
  \item Bacground on cardinality of columns
  \item NUMA
  \item Selectivity
\end{enumerate}
\end{secex}
\clearpage

\section{Background on Business Intelligence Software}
\label{sec:bi-background}
\textit{For managers to know what information they need (1) they must be aware of each type of decision they should make and (2) they must have an adequate model of each. The second condition, if not the first, is seldom satisfied. The genius of a good manager lies in his ability to manage effectively a system that he does not understand completely. It has long been known in science that the less we understand something, the more variables we require to explain it. Therefore, the manager who is asked what information he needs to control something he does not fully understand usually plays it safe and says he wants as much information as he can get. The MIS designer, who understands the system involved even less than the manager does, adds another safety factor and tries to provide everything. The result is an overload of information, most of which is irrelevant. The greater this overload, the less likely a manager is to extract and use whatever relevant information it contains. The moral is simple: one cannot specify what information is needed for decision making until a valid explanatory model of the decision process and the behavior of the system involved has been constructed.}


\subsection{The history of business intelligence}


\subsection{Challenges with the current solutions}
The current solutions in business intelligence lacks a process support.

The soultions of today lacks a two-way binding to the data, and relies on preaggregated dataware-houses. Occasionally, the analytics boils down to a small selection of objects (e. g. employees, or transactions), but there is no way to navigate to the OLTP part of the application again.

They are integrated in the way they talk together through protocols and nightly batches, but they are not integrated on a top level. They are separate products that needs to be installed and maintained.

Lastly, they don't use metadata at the highest level.


\section{\bd}
\label{sec:Business Discovery}
The challenges with data warehouses is many, among them, the lack of flexibility. Reports are normally preconfigured by the IT department, and this can result in long reporting backlogs. To counter these problems, a new type of products have emerged. We call these products \bd~products, a notion that was introduced by \qlikview \cite{Qlik2014-vd}. Examples of these products are \powerpivot, \tableau, and \qlikview. Studied in this thesis are the two latter, and we have denoted these as our reference products.

\ffigure{img/qlik-hierarchy.png}{Comparison of a traditional reporting application and \qlikview. Courtesy of \cite{Qlik2014-vd}.}{fig:qlik-hierarchy}
\bd~products allow userss to follow ther "information scent" or "train of though" \cite{Qlik2014-vd, Kamkolkar2015-iq}. There are no particular drill-down patterns that are prespecified, users might navigate through the data as they see fit, as seen in Figure~\ref{fig:qlik-hierarchy}.

We mainly focus on \qlikview~when explaining how a typical \bd~application work. \tableau~and \powerpivot~work similarly. We explain how \bd~applications work here since these products are the main specification for the new functionality in \genusSoftware.

\subsection{How these applications work}
\label{sub:How these applications work}
\ffigure{img/qlik-panel.png}{\qlikview~dashboard with various GUI elements. Selections are marked green, matched data remains white, but unrelated data is colored gray. Courtesy of \cite{Qlik2014-vd}.}{fig:qlik-panel}
Users interact with the \bd~application through a reporting dashboard, as seen in Figure~\ref{fig:qlik-panel}. Requirements for this panel are typically, but not limited to \cite{Qlik2014-vd}:
\begin{itemize}
  \item Clicking field values in list boxes.
  \item Lassoing data in charts and graphs and maps.
  \item Manipulating sliders.
  \item Choosing dates in calendars.
  \item Cycling through various charts.
\end{itemize}

In the reporting panel, \qlikview~follows a simple coloring strategy, as seen in Figure \ref{fig:qlik-panel}. Green means selected, white means matched, and gray means unmatched. The reason for displaying unmatched results is that business insight requires knowledge of not only what is there, but also what is missing. The green selections indicates the \term{application state}.

Distinct for the \bd~applications is that they do not favor any particular queries or drill-down paths. Hence, although results might be cached, the data is not preaggregated. Calculations and joins are calculated as needed.

To design a \bd~panel, a data extract layer must be defined. In \qlikview, this means configuring tables and relations through an import script. The import script uses a language similar to SQL in order to define tables, relations, and import statements. Connections various data sources are defined in this script, and it supports various formats and database systems. When the data import script is run, data is fetched from the sources and put into the \qlikview~in-memory engine such that data can be queried efficiently. \tableau~works similarly, however, in addition to the built in in-memory engine, the application allows for connecting directly to database servers \cite{Kamkolkar2015-iq}. This way, the users can utilize investments in high-performance databases. 

In the data import script, data might be transformed before loaded into the in-memory engine. Here, data can be filtered, summarized and grouped. One can also denormalize the data by joining one or more tables during the import. We look into denormalization in Section \ref{sec:Denormalization}.

A requirement for the data import script, is that the data follows a snowflake schema. That is, if a table is used in multiple relations, the table itself will be duplicated. This ensures that there exists no more than one possible join path between any pairs of tables \cite{noauthor_undated-js}. \todo{Find out that this actually is the definition of a snowflake schema, or another term must be used}

Both \qlikview~and \tableau~publish the data through a server \cite{Kamkolkar2015-iq, Qlik2011-ef}. This way, the dashboards are accesible for multiple users at the same time, and can be accessed from different devices, like tablets and mobile phones.

%In terms of metadata management, both IT and business employees are empowered to modify it.

\subsection{Queries}
\label{sub:Queries}
We see that queries in \bd~applications have certain distinctions. First of all, the queries are ad-hoc, they cannot be anticipated. Second, not many elements are returned, since the GUI is limited in how many distict elements that can be shown at the same time. \todo{Find the source on this} 

It is worth noting that \bd~applications do not need to support all SQL operations which a DBMS normally does. For instance, joins can be limited to equi-joins over foreign keys, and aggregations may be limited to groups of one or more dimensions in the schema.

\subsection{Performance}
\label{sub:Performance}
The main selling point for the \bd~products is that they perform well, but several factors contribute to the system overall performance. When considering performance, it is done from an end-user perspective \cite{Qlik2011-yc}.

\qlikview~outlines four aspects of application performance: Size of data, number of users, number of applications, and application design. The latter aspect means you should avoid too many listboxes, too many tables, too many formulas. In addition to this, columns should be split for better compression, and only pick the fields that are needed.

In other words, not only designing a high performance in-memory backend is important for the user experience; \textbf{much can be done in the application design}. Data can be filtered or summarized during import. An example of this is a business discovery panel for a retail store company, where the highest granularity is per day. Here, transactions should be summarized in the import script in order to reduce data size.
\subsection{Security settings}
\label{sub:Security settings}

The main \qlikview~security setting is the document-level authorization \cite{Qlik2011-hj}. Using this technique, certain documents, or data extracts, are restricted to certain users. This way, several data marts with different data extracts can be served to the various user groups. In addition to this, a dynamic filter can be applied to the data, such that new data marts do not need to be defined for every use case. Fields an columns can be removed using a special \texttt{OMIT} keyword.

As for \tableau, data-level and user-level security is supported \cite{Kamkolkar2015-iq}. Data can be filtered per user as long as the filter matches a where-clause. Like \qlikview, \tableau~can limit the access to different reports and views.

\saph~has special support for analytical privileges that filters the data or applies drill-down limitations \cite{Primsch2011-ij}.

\subsection{Challenges with the current solutions}
The current solutions in business intelligence lacks a process support.

The soultions of today lacks a two-way binding to the data, and relies on preaggregated dataware-houses. Occasionally, the analytics boils down to a small selection of objects (e. g. employees, or transactions), but there is no way to navigate to the OLTP part of the application again.

They are integrated in the way they talk together through protocols and nightly batches, but they are not integrated on a top level. They are separate products that needs to be installed and maintained.

Lastly, they don't use metadata at the highest level.

%\section{In-Memory Database Systems}
\label{sec:imds}
According to \cite{Zicari2012-is}, an in-memory database system is a system that uses memory (RAM) as its primary storage, and requires no disk or I/O operations. In-Memory database systems can be used where performance and low latency is a key design goal, and on systems that simply has no need for persistant storage.

In-memory databases strive to use the storage efficiently, since memory is not abundant as ordinary hard drives.

%\section{Column stores}
The past decade witnessed a lot of columnar storages used for business analatics. These databases are normally placed in a dataware house, and updated every night. Instead of saving data row-wise, which is handy for transactional data, column stores stores the data in columns such that aggregations can be calculated more easily.

One of the first major column store DBMS was the C-Store \cite{Stonebraker2005-qz}, developed my Stonebraker et. al. This database uses...


\section{Data skew}
\label{sec:Data skew}
It is quite common that data is skewed. Typically, for a retailer, 99\% of the sales is done on weekdays, 40\% around christmas \cite{Raman2008-gi}. Although said to improve cache locality \cite{Larson2013-mc}, it is important that structures, compression schemes, and algorithms takes data skew into account. An example of this is \term{PForDelta}, which is a compression that builds on bit packing \cite{Bjorklund2011-wh}. In this scheme, outliers are not bit packed.


\section{Testing analytical workloads}
\label{sec:Testing analytical workloads}
To test analytical workloads, it is common to use the TPC-H benchmark \cite{Boncz2002-yj}. This is used for ad-hoc querying. To test for large data-sets, a scaling factor (SF) can be used. The TPC-H query focus on expression calculation, and consists of queries that are both memory and CPU bound \cite{Boncz2005-wj}.

Of other benchmarks, we see Star Schema Benchmark \cite{Boncz2002-yj}

When testing, different selectivities and data distributions should be used. Selectivities may vary from 0.01\% to 50\%, and data should be tested with both uniform and skewed distributions. Data skew can be modelled with a \term{Zipfian} distribution \cite{Holloway2008-rr}.

As a practical test for how much memory and CPU you need, the \qlikview~developers suggest to deploy a smaller scale deployment, and measure resource usage and do a linear extrapolation \cite{Qlik2011-yc}.

Other ways to test the implementation, includes using \pn{callgrind} to observer branches and cache effects \cite{Neumann2011-uq}, \pn{Intel Performance Counter Monitor} to measure cycles and nanoseconds \cite{Willhalm2013-ri}, and \pn{JMeter} to test user interaction \cite{Qlik2012-ku}.


