\section{\bi}
\label{sec:bi-background}
\bi~is normally described as tools and techniques used to transform unstructured data into useful and meaningful information that can be used to support decisions \cite{Wikipedia_contributors2015-ag}. The goal is to allow for easy interpretation and gain insight in the data, such that businesses end up with a competitive market advantage.

In order to know which information is needed in a \bi~application, decision makers must be aware of which types of decicions they should make, and have a model for each \cite{x}. Since managers rarely fufils the latter requirement, they add a safety factor and asks the IT department to provide everything. The result is information overload, and the much of the data is irrelevant. \todo{Find source for this}

%\textit{For managers to know what information they need (1) they must be aware of each type of decision they should make and (2) they must have an adequate model of each. The second condition, if not the first, is seldom satisfied. The genius of a good manager lies in his ability to manage effectively a system that he does not understand completely. It has long been known in science that the less we understand something, the more variables we require to explain it. Therefore, the manager who is asked what information he needs to control something he does not fully understand usually plays it safe and says he wants as much information as he can get. The MIS designer, who understands the system involved even less than the manager does, adds another safety factor and tries to provide everything. The result is an overload of information, most of which is irrelevant. The greater this overload, the less likely a manager is to extract and use whatever relevant information it contains. The moral is simple: one cannot specify what information is needed for decision making until a valid explanatory model of the decision process and the behavior of the system involved has been constructed.}


\subsection{History}
There is a need for analysis in buisness trends, and normally business consists of legacy systems ussing a large number of different OLTP databases.

Later came the decision support systems, which naturally leaded to a data warehouse. A data warehouse is a system specifically designed for OLAP workloads, where data is stored in nightly batch jobs, and preaggregated such that the data can be analysed the next day. Later, the notion of data marts was introduced, where a data mart is a portion of a data warehouse that is relevant for a specific department.

On the way toward the \bd~products we know today, came the Operational Data Store (ODS)~\cite{Pavlic2002-nm}. This is another architectural construct that comes in addition to the OLTP legacy systems and the OLAP data warehouses. An ODS is created for efficient data extraction without affecting OLTP. This can be seen as a first step toward analytics on live data.

Recently, work has been put into uniting OLAP and OLTP systems into OLXP, such that the same system can be used for both transactional systems and business intelligence. Systems here include \hyper~and \hyrise, which offers high performance for OLXP workloads. \todo{Elaborate + sources}

\subsection{Star and Snowflake Schema}
\label{sub:Star and Snowflake Schema}
Often used in \bi~applications and data warehouses, are the star or snowflake schemas. This schema distinquishes between dimension tables, which is normally small, and a fact table, which is normally large. The fact table is then connected with one of more dimensions into a schema that looks like a star. A snowflake schema is an extension of the star schema, where dimensions can have more dimensions. 

Distinct for star and snowflake schemas is that there is only one distinct path through the schema. In other words, if the fact table relates to a dimension table through several fields (like \textit{CreatedByUser}, \textit{ModifiedByUser}), the dimension table occurs multiple times in the schema, even though it is the same logical table. \todo{Check if this is correct}

Queries on star and snowflake schemas are easier to optimize \cite{Lamb2012-kg} \todo{More info}. Techniques has been developed that lets you turn a non-star schema into a star or snowflake to help the query optimizer find efficient execution plans.

