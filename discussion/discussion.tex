\chapter{Discussion}
\label{chap:discussion}

\section{Future work}
Although this thesis has discussed the key challenges on implementing Business Intelligence in a model-driven framework, there are still challenges that has not been discussed broadly.

The thesis has assumed that data traverse the network at low cost, such that all the data for a specified query is available on the client. However, with large amounts of data, as well as slower network with restrictions, this might not be the case. To get high performance, values should be fetched "on demand". In addition, the client should be able to handle delta updates, such that the whole data set is not refreshed on every update.

The thesis is based on a model where the data is recycled every night. This means that the most recent data will be left out of the analysis. Here it would be interesting to find out which alterations that needs to be done to avoid this problem. One could look at bla bla.

There is also a lack of discussion of which component should do what. As explanied, \genusSoftware is comprised of a server and a client. Moving functionality between these components will affect the latency and performance.
