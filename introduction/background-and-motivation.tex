\section{Background and Motivation}
\label{sec:Background and Motivation}

\bi, or \ba, allows for making qualified decisions based on data gathered mainly in the IT systems within a company. Traditionally, data warehouses have been used for this, with reports that are preconfigured and generated in batches. However, with the advent of in-memory databases and cheap, commodity hardware, several market players have come up with fast, elegant, and end user intuitive solutions to analyze business data. Products like \qlikview, \tableau, and \powerpivot~offer high-performance analytics panel \todo{panel kom litt brått} that are easy to configure for the end user, such that important patterns are spotted and acted upon. These products help people answer their stream of questions, and can follow their own path to business insight. We refer to these as \term{Business Discovery} products \cite{Qlik2014-vd}.

There are several challenges with current \bd~products. First of all, they are separate products and do not integrate well with existing solutions. Second, these applications require explicit knowledge about metadata, more specifically; types and database relations must be reconfigured when defining the data import routine.  Third, to handle multiple users at once, these systems require a separate system architecture, which is yet another system for the IT staff to install and maintain. In general, the \bd~products work in isolation and are detached from the underlying data.

\genus, a Norwegian software provider, develops a model-driven application framework that enables development of software applications without programming. \genusSoftware, their main product \todo{har bare et produkt}, allows expert users to create a system business model with relations, rules, and restrictions. In combination with action orchestration and form design, a complete IT system \todo{ulik begrepsbruk} for a company can be made. Applications created in \genusSoftware~are deployed on a server and is accessible using desktop clients and mobile devices.

By implementing \bd~capabilities in the \genusSoftware, the challenges in existing \bd~products can be overcome. If the functionality is implemented correctly, users can access \bd~dashboards and data extracts that seamlessly interacts with their main IT system. Object relations and security settings are kept, and the same server architecture can be used. 

An early stage of \bd~capabilities in \genusSoftware~have already been implemented by \genus~and released to certain test customers. However, the implementation is not tailored for analytical queries, and it performs poorly compared to other \bd~products. Nor does it support large amounts of data. 

The motivation for this thesis is, therefore, to improve the \bd~performance in \genusSoftware. To do this, we study relevant literature on \bi, online analytical processing workloads, and look at other \bd~products to identify key concepts and challenges in implementing such a system. 
