\section{Background and Motivation}
\label{sec:background-and-motivation}
%\begin{secex}
%    This section should introduce:
%    \begin{enumerate}
%      \item \genus
%      \item The need for Business Intelligence
%      \item The fact that \genus~needs to offer business intelligence to offer a competitive product
%      \item Problems with existing products, and they believe \genus~can offer a better product.
%    \end{enumerate}
%\end{secex}

\genus~is a mid-sized Norwegian company that creates a model-driven application framework that lets you build software applications without programming. The framework lets you define your business object model and configure restrictions and relationsships within the model. Using action orchestration and form design, \genusSoftware~can provide a fully functional development framework that can solve a magnitude of problems. It handles millions of rows. \genusSoftware~does also include restrictions on access, which means the data available to the end users in the model is dependent on the user group. \todo{A couple of more sentences on the \genus company and software}

\bi, or \ba, is an important piece of a business. Business Intelligence allows for making qualified decisions based on data gathered mainly in the IT systems within the company. Traditionally, data warehouses (section X) has been used for this, with reports are preconfigured and generated in batches. However, with the advent of in-memory databases and cheap, commodity hardware, several market players has come up with fast, elegant, and end user intuitive solutions to analyze business data. Products like \qlikview, \tableau, and \powerpivot~offers high-performance analytics panel which are easy to configure for the end user, such that important patterns are spotted and acted upon. These products help people answer their own stream of questions, and can follow their own path to business insight. We refer to these as \term{Business Discovery} products \cite{Qlik2014-vd}.

There are several challenges with these \bd~products. First of all, they are mostly separate products, and does not integrate well with existing solutions. This affects the end users, who will have to install and familiarize themselves with a new and unfamiliar user interface. The IT department does also need to specify a separate security system for these applications, such that the data is only available for authorized users. Second, these applications require explicit knowledge about metadata, more specifically; types and database relations needs to be reconfigured when defining the data import routine. Third, in order to handle multiple users at once, these systems require a separate architecture, which is yet another system for the IT staff to install and maintain. In general, the \bd~products work in isolation and are detached to the underlying data.

By implementing \bd~capabilities in the \genusSoftware, the above challenges can be overcome. If implemented correctly, users is able to access \bd~dashboards and data extracts that seamlessly interacts with their main IT system. Security settings and metadata is kept, and the same server architecture can be used. 

An early stage of this system has already been implemeted by \genus~and released to certain test customers. This system uses columnar storage and bitmap indexes, and the data is put entirely into memory. However, much of the implementation is not tailored for analytical queries, and it performs poorly compared to other \bd~products, and it does not support large amounts of data. The motivation for this thesis is hence to improve the \bd~performance in \genusSoftware. To do this, we study the relevant litterature in Business Intelligence and online analytical processing workloads to identify key concepts and challenges in implementing such a system. When such concepts, techniques and challenges are identified, we conclude the paper with which techniques are the most promising and practical in \genus' situation.
