\section{Problem Statement, Goals, and Deliverables}
\label{sec:problem-statement-and-goals}
Based on our motivation, we define the goal of the project as following: 

\textbf{G1: Implement \bd~capabilities in \genusSoftware~ that has high performance, handles large datasets, and utilizes the available hardware. The new product must be competitive to other \bd~products regarding performance and functionality.}

By \bd~products, we mean high-performance systems that allow for ad-hoc querying and data navigation in all dimensions without relying on preaggregation of data. Examples of such products are \qlikview, \tableau, and \powerpivot. We study \bd~products in greater detail in Section \ref{sec:Business Discovery}.

We define \textit{competitive to other \bd~products in terms of performance and fuctionality} as following:
\begin{itemize}
  \item We compare performance directly to alternative \bd~products. Here, we mainly consider application response time for end users, but we also consider memory footprint.
  \item Regarding functionality, we only consider essential functionality in the \bd~products that helps end users to obtain business insight. Essential functionality includes support for a configurable GUI with a variety of elements and graphs, in addition to selections and filters.
\end{itemize}

We have defined this goal because we are aware that \genus' customers will compare the \bd~capabilities in \genusSoftware~with alternative products. 

As a first step towards reaching \textbf{G1}, we address the following research question:

\textbf{RQ1: How to design a high-performance database software that is capable of supporting \bd~workloads on large datasets?} 

We answer \textbf{RQ1} by reviewing the current literature on \bi, online analytical workload (OLAP) systems, and \bd~products.
