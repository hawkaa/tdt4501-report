\section{Methodology}
\label{sec:Methodology}
\todo{Correct and enhance section}
The main part of the literature review was conducted using a method known as \term{Snowballing}, which is convenient if the scope of the project is uncertain \cite{Ang2014-nm}. The Snowballing method is the process whereby you start with a few number of authorative papers, and based on these you expand your list of readings by relevant work which the papers have cited. The identification of papers can also happen in the other direction, where you look for papers that have used the current one as a reference. Either way, this method is know to generate a large number of papers, so the researcher must be very strict in which papers to read. Each paper should be objectively reviewed, as picking articles subjectively can result in a biased set of readings.

The initial papers, theses, and books used in this research was found in collaboration with department staff, and regarded in-memory databases, columnar storage, and online analytical processing (OLAP) workloads. Both forward and backward searching was performed, and each paper was considered by reading the abstract, and conclusion and intrudction if needed, to be put on the reading list. During the search process, we picked articles that could help us reach \textbf{G1} and answer \textbf{RQ1}. For this part of the research, all literature must have been published by a known digital library, journal, or conference. We concluded the literature study when we the field was well understood, and when we already had read the majority of the relevant references in the residiual papers.

The above process yielded a variety of sources relevant to database management systems (DBMS) and OLAP workloads. However, in the snowballing process, we did not find two types of information. First, we felt our understanding of state-of-the-art commercial database technologies was missing. Second, and perhaps most important, we had not found any litterature specific to \bd~products. Hence, in addition to performing a \term{Snowball} litterature review mentioned above, we scanned the websites of products identified in the previous step and previously by \genus. Here, we looked for whitepapers, blog posts, articles, and forum posts that could reveal how these products solves the problem. Products here include \pn{QlikView}, \pn{Tableau}, \pn{Microsoft SQL Server}, \pn{Oracle Database}, and \pn{EXASOL}.
