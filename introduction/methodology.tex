\section{Methodology}
\label{sec:Methodology}
\begin{secex}
  Will explain how I performed the literature review, and perhaps also a figure of the research process. Can read Marte's text for motivation here
\end{secex}

The main part of the literature review was conducted using a method known as \term{Snowballing}, which is convenient if the scope of the project is uncertain \cite{Ang2014-nm}. The \term{Snowballing} method is the process whereby you start with a few number of authorative papers, and based on these you expand your list of readings by relevant work which the papers have cited. The identification of papers can also happen in the other direction, where you look for papers that have used the current one as a reference. Either way, this method is know to generate a large number of papers, so the researcher must be very strict in which papers to read. Each paper should be objectively reviewed, as picking articles subjectively can result in a biased set of readings.

The initial papers, theses, and books used in this research was found in collaboration with my supervisor, and regarded in-memory databases, columnar storage, and OLAP workloads. Both forward and backward searching was performed, and each paper was considered by reading the abstract, and conclusion and intrudction if needed, to be put on the reading list. Througout the search process, literature regarding OLAP performance was picked. For this part of the research, all literature must have been published by a known digital library, journal, or conference. 

The literature study ended when I felt I had good understanding of the problem, and when I already had read the majority of the relevant references in the residiual papers.

In addition to performing a \term{Snowball} litterature review mentioned above, I scanned the websites of products identified in the previous step and previously by \genus. Here, I looked for whitepapers, blog posts, articles, and forum posts that could reveal how these products solves the problem. Products here include \pn{QlikView}, \pn{Tableau}, \pn{Microsoft SQL Server}, \pn{Oracle Database}, and \pn{EXASOL}.
