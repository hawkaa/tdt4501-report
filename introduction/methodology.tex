\section{Methodology}
\label{sec:Methodology}
The main part of the literature review was conducted using a method known as \term{Snowballing}, which is convenient if the scope of the project is uncertain \cite{Ang2014-nm}. The Snowballing method is the process whereby you start with a few number of authoritative papers, and based on these you expand your list of readings by relevant work that the papers have cited. The identification of papers can also happen in the other direction, where you look for papers that have used the current one as a reference. Either way, this method is known to generate a large number of papers, so the researcher must be very strict and objective in which papers to read.

The initial papers, theses, and books used in this research were found in collaboration with department staff and regarded in-memory databases, columnar storage, and online analytical processing (OLAP) workloads. Both forward and backward searching was performed, and each paper was considered by reading the abstract, and conclusion and introduction if needed, to be put on the reading list. During the search process, we picked articles that could help us reach \textbf{G1} and answer \textbf{RQ1}. Also, articles must have been published by a known digital library, journal, or conference. 
When the field felt properly understood, we concluded the \term{Snowball} literature study.

The above process yielded a variety of sources relevant to database management systems (DBMS) and OLAP workloads, but we failed to find literature about state-of-the-art commercial database technologies and \bd~products. Hence, in addition to performing a \term{Snowball} literature review, we also scanned the websites of products identified in the previous step. We also looked into products identified in preliminary research by \genus. We searched for whitepapers, blog posts, articles, and forum posts that could reveal how these products answer our research question. Products here include \pn{QlikView}, \pn{Tableau}, \pn{Microsoft SQL Server}, \pn{Oracle Database}, and \pn{EXASOL}.
