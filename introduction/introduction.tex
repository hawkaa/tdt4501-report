\chapter{Introduction}
\label{chap:introduction}
\clearpage

\section{Background and Motivation}
\label{sec:background-and-motivation}

\bi, or \ba, allows for making qualified decisions based on data gathered mainly in the IT systems within a company. Traditionally, data warehouses have been used for this, with reports that are preconfigured and generated in batches. However, with the advent of in-memory databases and cheap, commodity hardware, several market players has come up with fast, elegant, and end user intuitive solutions to analyze business data. Products like \qlikview, \tableau, and \powerpivot~offers high-performance analytics panel that are easy to configure for the end user, such that important patterns are spotted and acted upon. These products help people answer their stream of questions, and can follow their own path to business insight. We refer to these as \term{Business Discovery} products \cite{Qlik2014-vd}.

There are several challenges with current \bd~products. First of all, they are separate products and does not integrate well with existing solutions. Second, these applications require explicit knowledge about metadata, more specifically; types and database relations must be reconfigured when defining the data import routine.  Third, to handle multiple users at once, these systems require a separate system architecture, which is yet another system for the IT staff to install and maintain. In general, the \bd~products work in isolation and are detached from the underlying data.

\genus, a Norwegian software provider, develops a model-driven application framework that enables development of software applications without programming. \genusSoftware, their main product, allows expert users to create a system business model with relations, rules, and restrictions. In combination with action orchestration and form design, a complete IT system for a company can be made. Applications created in \genusSoftware~are deployed on a server and is accessible using desktop clients and mobile devices.

By implementing \bd~capabilities in the \genusSoftware, the challenges in existing \bd~products can be overcome. If the functionality is implemented correctly, users can access \bd~dashboards and data extracts that seamlessly interacts with their main IT system. Object relations and security settings are kept, and the same server architecture can be used. 

An early stage of \bd~capabilities in \genusSoftware~have already been implemented by \genus~and released to certain test customers. However, the implementation is not tailored for analytical queries, and it performs poorly compared to other \bd~products. Nor does it support large amounts of data. 

The motivation for this thesis is, therefore, to improve the \bd~performance in \genusSoftware. To do this, we study relevant literature on \bi, online analytical processing workloads, and look at other \bd~products to identify key concepts and challenges in implementing such a system. 

\section{Problem statement and goals}
\label{sec:problem-statement-and-goals}
In order to aid \genus~with the software development process, this thesis will address the following research question:

\textbf{RQ1: Which challenges does \genus~face when implementing Business Intelligence support in their \genusSoftware?}

The question will be answered by reviewing the current litterature, scrutinizing the competing products and analysing the bits and bolts needed to develop business intelligence inside \genusSoftware.

Based on some workshops and discussions with \genus , they have pointed out some key issues they consider important. Based on these key points, I've come up with the following hypotheses:
\begin{enumerate}\bfseries
  \item Data needs to be accessed fast. To do this, data has to be stored in a format for quick access, and indexes has to be generated dynamically.
  \item The security layer in \genusSoftware~will affect the implementation, and some indexes and formula results will have to be stored per user.
  \item Indexes, data, and data interchange formats must be compressed to get performance and utilize the available memory efficiently.
  \item Formulas must be calculated efficiently to not become a bottleneck.
  \item \sout{Network traffic must be reduced by compressing data queries.}
  \item \sout{In order to create a high-performance system without a database administrator, indexes have to be created dynamically, and there is a question of which and when.}
\end{enumerate}

Each hypthesis will be studied in the terms of research, competitor solutions, as well as design proposal. In addition to this, I will look for unindentified issues which will affect the design.




\subsection{In-Memory}
\label{sub:In-Memory}
We restrict our research to databases that assumes the entire dataset can fit in RAM The reason for this is two-fold. First of all, preliminary research have shown that most database systems use this approach in order to achieve good performance. These systems include, but are not limited to, \oracle~\cite{Lahiri2015-mz}, \saph~\cite{Farber2012-vh}, \gorilla~\cite{Pelkonen2015-ko}, \qlikview~\cite{Qlik2011-ef}, \tableau~\cite{Kamkolkar2015-iq}, \monetdb~\cite{Boncz2002-yj}, \blink~\cite{Barber2012-xt}, and \sapnw~\cite{Lemke2010-is}. Several of these systems requires the database to fit entirely in memory, and does not have a buffer manager. We see later in this thesis that omiting the buffer manager can increase performance since an extra layer of indirection is removed \cite{Graefe2014-ds}, and it has been shown that databases without buffer managers performs better than those who do \cite{Ferrari2012-hm}. If the database gets to large than the provisioned RAM, data will be spilled to disk using the OS underlying swapping mechanisms.  A white paper by \qlikview~suggests that companies looking for business intelligence should look for in-memory technologies \cite{Bereanu2010-tj}.

Kemper \ea~arguest that it is safe to assume data can fit in memory, because of compression and price of RAM... \cite{Kemper2011-ap} \todo{Fill in information from Kemper here}. RAM is getting cheaper \cite{Exasol2014-xh}, and together with 64-bits CPUs \cite{Delaney2014-ip}, we find in-memory databases to outperform traditional disk-based databases. Much progress has been done in the development of non-volatile RAM, which suggests the era of magnetic disks as the primary database storage hardware might soon be over.

It has been said that putting entire databases in RAM is the key to unite OLAP and OLTP \cite{Faust2015-ke}, even though we do not emphasize write support in this thesis. It is also easier optimized using parallelization \cite{Psaroudakis2013-fn}.

It is worth noting that even though we focus on in-memory performance, optimizations performed for disk-based databases will still be applicable. We have just moved one step up through the memery hierarchy \cite{Boncz2002-yj}. That is, optimizations done to better utilize available memory for a disk based database will help an in-memory database utilize available CPU caches. \todo{Make this much clearer}

We will continue this research keeping the in-memory assumption true. However, even though DRAM is cheap, it is still rarely over-provisioned and unused \cite{Barber2014-ey}. We will try to keep the memory footprint as low as possible.

\subsection{Read-only}
\label{sub:Read-only}
In the first iteration towards the main goal, we assume that the system will be read only. By not supporting inserts, updates, and deletes, we simplify the database design. We will reason in Chapter x of which techniques might be applied if we need write support.

\ffigure{img/compromise.png}{Conceptual figures of how we expect the performance of mixed workloads to be affected by (a) data freshness, (b) flexibility, and (c) scheduling. Courtesy of \cite{Psaroudakis2014-ma}.}{fig:compromise}
We choose to focus on read-only because our main design goal is performance. A research paper by Psaurodakis \ea~\cite{Psaroudakis2014-ma} explains how "one size does not fit all" in a database setting, and in order to get good read performance, data freshness, query flexibility, and query scheduling must be compromised.

Write support comes in two levels. The first, and most involved level is to have the database system support inserts, updates, and deletes directly. These systems might also support transactions. Typical systems are DBMS, where the focus is correctness, and the systems might emphasize read performance, write performance or both. The other, more light-weight alternative, is to have the \bd~solution listen to updates performed by another system, say a DBMS. These solutions are not guaranteed to be timely, but timely enough for most cases. \qlikview~and \tableau~supports the latter.

%\section{Deliverables}
\label{sec:Deliverables}
\begin{secex}
This part will explain the two main deliverables: A thorough litterature review, and some indication on which techniques are the most fruitful for \genusSoftware.
\end{secex}
The first two parts of the thesis will be a literature review and a similar product analysis. These two chapters let me familiarize myself to the most recent research within the field, and have a look at how competing products solves the problems. The ultimate goal for these chapters is to be able to come up with a discussion and analysis of the hypotheses stated in section \ref{sec:problem-statement-and-goals}, which will be the third pard of the thesis. This part will also include a section of other issues that has not been thought of so far.

The last part of the thesis, based on the results so far, will contain a design proposal for a Busines Intelligence implementation in \genusSoftware. 

\section{Methodology}
\label{sec:Methodology}
\begin{secex}
  Will explain how I performed the literature review, and perhaps also a figure of the research process. Can read Marte's text for motivation here
\end{secex}

The main part of the literature review was conducted using a method known as \term{Snowballing}, which is convenient if the scope of the project is uncertain \cite{Ang2014-nm}. The \term{Snowballing} method is the process whereby you start with a few number of authorative papers, and based on these you expand your list of readings by relevant work which the papers have cited. The identification of papers can also happen in the other direction, where you look for papers that have used the current one as a reference. Either way, this method is know to generate a large number of papers, so the researcher must be very strict in which papers to read. Each paper should be objectively reviewed, as picking articles subjectively can result in a biased set of readings.

The initial papers, theses, and books used in this research was found in collaboration with my supervisor, and regarded in-memory databases, columnar storage, and OLAP workloads. Both forward and backward searching was performed, and each paper was considered by reading the abstract, and conclusion and intrudction if needed, to be put on the reading list. Througout the search process, literature regarding OLAP performance was picked. For this part of the research, all literature must have been published by a known digital library, journal, or conference. 

The literature study ended when I felt I had good understanding of the problem, and when I already had read the majority of the relevant references in the residiual papers.

In addition to performing a \term{Snowball} litterature review mentioned above, I scanned the websites of products identified in the previous step and previously by \genus. Here, I looked for whitepapers, blog posts, articles, and forum posts that could reveal how these products solves the problem. Products here include \pn{QlikView}, \pn{Tableau}, \pn{Microsoft SQL Server}, \pn{Oracle Database}, and \pn{EXASOL}.


\section{Contributions}
\label{sec:Contributions}
The novelty of this thesis is two-fold. Most related work presesnts a single system and explains the techniques use there. Here, we present the litterature by category, not by product. This way, comparisons can be made more easily, and simplifies the process to getting an overview over where the litterature agrees, and where there are contradictions. We have found work X and work Y. These systems present single concepts, but neither of these span as broadly as we do in this thesis.

The second contribution, is a thorough discussion on where the various techniques presented applies. \genusSoftware have several fundamental and practical limitations, and based on these, We present the techniques that are the most fruitful. Optimizations in columnar storage have already been investigated by Abadi \ea~\cite{Abadi2008-dd}, however this work focus on the fundamental differences between row and column stores. We present a more holistic picture, considering more concepts and techniques. In addition to this, this thesis specifically targets \genusSoftware.

\section{Terms used in this thesis}
\label{sec:Terms used in this thesis}
This section presents terms and definitions used throughout this thesis.

\paragraph{Database Management System (DBMS)}
\label{par:Database Management System (DBMS)}
We look into several Database Management Systems (DBMS) in this thesis. These systems are general purpose systems for storage of data. DBMSes can focus on read performance for analytical workloads, write performance for transactional workloads, or both. These systems do not come with user interfaces for \bd, but is designed such that other applications can be built on top of them. The normal interface with the DBMSes is SQL. In this thesis, we look at \oracle, \ibm, \saph, \sapnw, \mssql, \cstore, \vertica, \blink, \exasol, \oracle, \hyper, and \hyrise.

\paragraph{Business Discovery}
\label{par:Business Discovery}
\bd~is a term introduced by \qlikview~\cite{Qlik2014-vd}. \bd~is different from traditional \bi~by focusing more on the end user. By not relying on preaggregated data, the user can follow his own "information scent" and click his way through the data. \bd~platforms delivers panels and dashboards to multiple devices, and allows for easy sharing. They typically build on storage systems that are specifically designed for \bd~workloads, but some of them integrate directly with read-optimized DBMSes. \bd~products include \tableau, \qlikview, \powerpivot, and more.

\paragraph{Reference products}
\label{par:Reference products}
By reference products we mean product pointed out \genus, products which \bd~in \genusSoftware~will be directly compared to. Studyed in this thesis is \qlikview~and \tableau.


\section{Thesis Outline}
\label{sec:Thesis Outline}
In Chapter 2, we give an introduction to backgound material relevant to this thesis. In Chapters 3-7, we discuss our findings in the literature review by the categories Data Layout, Compression, Implementation, and Scaling \todo{maintain this list}. In Chapter 8, we explain that careful implementation is needed to get good in-memory query performance. In Chapter 9, we discuss other techniques and aspects that is related to performance. Chaper 10 concludes this thesis and points at interesting directions for future work.



