\chapter{Introduction}
\label{chap:introduction}
\clearpage

\section{Background and Motivation}
\label{sec:background-and-motivation}
\begin{secex}
    This section should introduce:
    \begin{enumerate}
      \item \genus
      \item The need for Business Intelligence
      \item The fact that \genus~needs to offer business intelligence to offer a competitive product
      \item Problems with existing products, and they believe \genus~can offer a better product.
    \end{enumerate}
\end{secex}

\genus~is a mid-sized Norwegian company that creates a model-driven application framework that lets you build software applications without programming. The framework lets you define your business object model and configure restrictions and relationsships within the model. Using action orchestration and form design, \genusSoftware can provide a fully functional development framework that can solve a magnitude of problems. It handles millions of rows. \genusSoftware~does also include restrictions on access, which means the data available to the end users in the model is dependent on the user group. \todo{A couple of more sentences on the \genus company and software}

Business Intelligence, or Business Analytics, is an important piece of a business. Business Intelligence allows for making qualified decisions based on data gathered mainly in the IT systems within the company. Traditionally, data warehouses (section X) has been used for this, with preconfigured reports. However, with the advent of in-memory databases and cheap, commodity hardware, several market players has come up with fast, elegant, and end user intuitive solutions to analyze business data. Products like \qlikview, \tableau, and \powerpivot~offers high-performance analytics panel which are easy to configure for the end user, such that important patterns are spotted and acted upon. \todo{Add relevant info about business intelligence and business discovery}

There are several challenges with these products. First of all, they are all separate products, and does not integrate well with existing solutions. This affects the end users, who will have to install and familiarize themselves with a new and unfamiliar user interface. The IT department does also need to specify a separate security system for these applications, such that the data is only available for authorized users. Second, these applications require explicit knowledge about metadata\todo{ask Einar about this}, more specifically; types and database relations needs to be reconfigured when defining the data import routine. Third, in order to handle multiple users at once, these systems require a separate architecture, which is yet another system for the IT staff to install and maintain. In general, \qlikview, \tableau, and \powerpivot~work is isolation and detached to the underlying data.

\genus~wants to offer Business Intelligence capabilities to their customers, because the above challenges cann be overcome by implementing this directly within their \genusSoftware. If implemented correctly, users is able to access Business Intelligence dashboards and data extracts that seamlessly interacts with their main IT system. Security settings and metadata is kept, and the same server architecture can be used.

The motivation for this thesis is to aid \genus~ in the process of adding Business Intelligence capabilities to their software. To do this, we study the relevant litterature in Business Intelligence and OLAP workloads to identify key concepts and challenges in implementing such a system. When such concepts, techniques and challenges are identified, we conclude the paper with which techniques are the most promising and practical in \genus' situation.

\section{Problem statement, goals, and deliverables}
\label{sec:problem-statement-and-goals}
\begin{secex}
This section should explain what I will to to aid \genus~in solving the above motivation. That is more specifically studying the litterature, and see which techniques are used to enhance OLAP performance, because none of us know too much about what is going on out there. It should state the research question, which should be pretty similar to the Monet thesis.
\end{secex}

For the entire Business Intelligence project, the following goal is defined:
\textbf{Implement Business Intellingence capabilities in \genusSoftware~ that has high performance, handles large datasets, and utilizes the available hardware. The new product must be competitive to it's reference products in terms of performance and fuctionality.}

Working towards this goal does not mean that the newly developed software must be equal in terms of functionality, it is more of a statement that Business Intelligence in \genusSoftware will be compared directly to alternative products, like \qlikview, \tableau, and \powerpivot. In terms of performance and "waiting" it is directly comparable, but lack of functionality in \genusSoftware will be directly be weighted against the benefits of having everything in one solution.

As a first step towards this goal, we address the following research question:
\textbf{RQ1: How to design a high performance database software that is capable of supporting ad-hoc Business Intelligence queries on large datasets?}

The question will be answered by reviewing the current litterature, scrutinizing the competing products, and analysing the bits and bolts needed to develop Business Intelligence capabilities within \genusSoftware. In this thesis, we familiarize ourselvese with the most recent research within the field, and have a look at how reference products solve the problems. 

Based on the results from the RQ1, we continue with the following research question:

\textbf{RQ2: Which of the techiques and concepts are the most promising and pracical to implement in \genusSoftware~in order to reach the goal, and which challenges are identified?}

This question will be answered by combining the results from RQ1 with a brief analysis of the \genusSoftware architecture into a discussion.

%Based on some workshops and discussions with \genus , they have pointed out some key issues they consider important. Based on these key points, I've come up with the following hypotheses:
%\begin{enumerate}\bfseries
%  \item Data needs to be accessed fast. To do this, data has to be stored in a format for quick access. Data access patters need to be analyzed and optimization for data retrieval must be done dynamically.
%  \item The security layer in \genusSoftware~will affect the implementation, and some indexes and formula results will have to be stored per user.
%  \item Indexes, data, and data interchange formats must be compressed to get performance and utilize the system resources efficiently.
%  \item \sout{Formulas must be calculated efficiently to not become a bottleneck.}
%  \item \sout{Network traffic must be reduced by compressing data queries.}
%  \item \sout{In order to create a high-performance system without a database administrator, indexes have to be created dynamically, and there is a question of which and when.}
%\end{enumerate}
%
%Each hypthesis will be studied in the terms of research, competitor solutions, as well as design proposal. In addition to this, I will look for unindentified issues which will affect the design.




\subsection{In-Memory}
\label{sub:In-Memory}
We restrict our research to databases that assumes the entire dataset can fit in RAM The reason for this is two-fold. First of all, preliminary research have shown that most database systems use this approach in order to achieve good performance. These systems include, but are not limited to, \oracle~\cite{Lahiri2015-mz}, \saph~\cite{Farber2012-vh}, \gorilla~\cite{Pelkonen2015-ko}, \qlikview~\cite{Qlik2011-ef}, \tableau~\cite{Kamkolkar2015-iq}, \monetdb~\cite{Boncz2002-yj}, \blink~\cite{Barber2012-xt}, \sapnw~\cite{Lemke2010-is}, and more. Several of these systems requires the database to fit entirely in memory, and does not have a buffer manager. We see later in this thesis that omiting the buffer manager can increase performance since an extra layer of indirection is removed \cite{Graefe2014-ds}, and it has been shown that databases without buffer managers performs better than those who do \cite{Ferrari2012-hm}. If the database gets to large than the provisioned RAM, data will be spilled to disk using the OS underlying swapping mechanisms.  A white paper by \qlikview~suggests that companies looking for business intelligence should look for in-memory technologies \cite{Bereanu2010-tj}.

Kemper \ea~arguest that it is safe to assume data can fit in memory, because of compression and price of RAM... \cite{Kemper2011-ap} \todo{Fill in information from Kemper here}. RAM is getting cheaper \cite{Exasol2014-xh}, and together with 64-bits CPUs \cite{Delaney2014-ip}, we find in-memory databases to outperform traditional disk-based databases. Much progress has been done in the development of non-volatile RAM, which suggests the era of magnetic disks as the primary database storage hardware might soon be over.

It has been said that putting entire databases in RAM is the key to unite OLAP and OLTP \cite{Faust2015-ke}, even though we do not emphasize write support in this thesis. It is also easier optimized using parallelization \cite{Psaroudakis2013-fn}.

It is worth noting that even though we focus on in-memory performance, optimizations performed for disk-based databases will still be applicable. We have just moved one step up through the memery hierarchy \cite{Boncz2002-yj}. That is, optimizations done to better utilize available memory for a disk based database will help an in-memory database utilize available CPU caches. 

We will continue this research keeping the in-memory assumption true. However, even though DRAM is cheap, it is still rarely over-provisioned and unused \cite{Barber2014-ey}. We will try to keep the memory footprint as low as possible.

\subsection{Read-only}
\label{sub:Read-only}
In the first iteration towards the main goal, we assume that the system will be read only. By not supporting inserts, updates, and deletes, we simplify the database design. We will reason in Chapter x of which techniques might be applied if we need write support.
\missingfigure{Figure on the three factors that must be compromised}
The first reason we chose to stick to read-only functionality, is that our main design goal is performance. A research paper by Psaurodakis \ea~\cite{Psaroudakis2014-ma} explains how "one size does not fit all" in a database setting, and there always must be a compromize between data freshness, flexibility, and scheduling. Of these three design goals, we decide scheduling (directly related to performance) is our main priority.

Second, we see that the main reference products, \qlikview~and \tableau, are read only solutions.


%\section{Deliverables}
\label{sec:Deliverables}
\begin{secex}
This part will explain the two main deliverables: A thorough litterature review, and some indication on which techniques are the most fruitful for \genusSoftware.
\end{secex}
The first two parts of the thesis will be a literature review and a similar product analysis. These two chapters let me familiarize myself to the most recent research within the field, and have a look at how competing products solves the problems. The ultimate goal for these chapters is to be able to come up with a discussion and analysis of the hypotheses stated in section \ref{sec:problem-statement-and-goals}, which will be the third pard of the thesis. This part will also include a section of other issues that has not been thought of so far.

The last part of the thesis, based on the results so far, will contain a design proposal for a Busines Intelligence implementation in \genusSoftware. 

\section{Methodology}
\label{sec:Methodology}
The main part of the literature review was conducted using a method known as \term{Snowballing}, which is convenient if the scope of the project is uncertain \cite{Ang2014-nm}. The Snowballing method is the process whereby you start with a few number of authoritative papers, and based on these you expand your list of readings by relevant work that the papers have cited. The identification of papers can also happen in the other direction, where you look for papers that have used the current one as a reference. Either way, this method is known to generate a large number of papers, so the researcher must be very strict and objective in which papers to read.

The initial papers, theses, and books used in this research were found in collaboration with department staff and regarded in-memory databases, columnar storage, and online analytical processing (OLAP) workloads. Both forward and backward searching was performed, and each paper was considered by reading the abstract, and conclusion and introduction if needed, to be put on the reading list. During the search process, we picked articles that could help us reach \textbf{G1} and answer \textbf{RQ1}. Also, articles must have been published by a known digital library, journal, or conference. 
When the field felt properly understood, we concluded the \term{Snowball} literature study.

The above process yielded a variety of sources relevant to database management systems (DBMS) and OLAP workloads, but we failed to find literature about state-of-the-art commercial database technologies and \bd~products. Hence, in addition to performing a \term{Snowball} literature review, we also scanned the websites of products identified in the previous step. We also looked into products identified in preliminary research by \genus. We searched for whitepapers, blog posts, articles, and forum posts that could reveal how these products answer our research question. Products here include \pn{QlikView}, \pn{Tableau}, \pn{Microsoft SQL Server}, \pn{Oracle Database}, and \pn{EXASOL}.


\section{Contributions}
\label{sec:Contributions}
The novelty of this thesis is two-fold. Most related work presesnts a single system and explains the techniques use there. Here, we present the litterature by category, not by product. This way, comparisons can be made more easily, and simplifies the process to getting an overview over where the litterature agrees, and where there are contradictions. We have found work X and work Y. These systems present single concepts, but neither of these span as broadly as we do in this thesis.

The second contribution, is a thorough discussion on where the various techniques presented applies. \genusSoftware have several fundamental and practical limitations, and based on these, I present the techniques that are the most fruitful. Optimizations in columnar storage have already been investigated by Abadi \ea~\cite{Abadi2008-dd}, however this work focus on the fundamental differences between row and column stores. I present a more holistic picture, considering more concepts and techniques. In addition to this, this thesis specifically targets \genusSoftware.

\section{Report Outline}
\label{sec:Thesis Outline}
We structure our report as following:
\begin{itemize}
  \item \textbf{Chapter 2} introduces background material relevant to this report. Here, we study \bi, \bd, and testing.
  \item \textbf{Chapter 3-6} go into details techniques that are paramount to OLAP and \bd~performance, namely \textit{data layout}, \textit{data compression}, \textit{indexes and auxiliary structures}, and \textit{parallelization}.
  \item \textbf{Chapter 7} studies techniques relevant for query processing in \bd~applications in terms of \textit{joining}, \textit{grouping and aggregation}, and \textit{query optimization}.
  \item \textbf{Chapter 8} shows that careful implementation is required for a system to perform well.
  \item \textbf{Chapter 9} elaborates on other considerations in a \bd~application, like \textit{application design}, \textit{denormalization}, \textit{disk support}, and \textit{write support}.
  \item \textbf{Chapter 10} concludes this report and points at interesting directions for future work.
\end{itemize}



