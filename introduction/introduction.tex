\chapter{Introduction}
\label{chap:introduction}

\section{Background on data warehouses}
\label{sec:background-data-warehouses}
\lipsum[3]

\section{Background on business intelligence solutions}
\lipsum[3]

\section{Motivation}
\genus~is a small Norwegian company that creates a model-driven application framework that lets you build software applications without programming. The framework lets you define your business object model. \genusSoftware~handles millions of rows. \genusSoftware~does also include restrictions on access, which means the data available to the end users in the model is dependent on the user group.

Business Intelligence has also become an important part of a business. Traditionally, data warehouses (section \ref{sec:background-data-warehouses}) has been used for this, with preconfigured reports. However, with the advent of in-memory databases with cheap hardware, several market players has come up with fast, elegant and end user intuitive solutions to analyze business data. Products like \qlikview, Foo, and Bar offers high-performance analytics panel which are easy to configure for the end user, such that important patterns are spotted and acted upon.

There are however several challenges with these products. First of all, they are all separate products, and does not integrate well with existing solutions. Secondly, they don't handle metadata. Types and database relations needs to be reconfigured on import. Thirdly, the products does not support security. Lastly, a database administrator is required.

\genus~has been requested by their customers to develop a module for business intelligence inside \genusSoftware . They believe the challenges mentioned in the previous paragraph can be overcome by integrating the module into their existing framework. This is because \genusSoftware~is already installed at the client and does not need any new software. In addition, metadata is well handled, as well as permission control. 

The motivation for this thesis is to aid \genus~ in the process of developing this module, by analysing state-of-the art research, identifying key issues and suggest design principles.
%\section{Background on Business Intelligence Software}
\label{sec:bi-background}
\textit{For managers to know what information they need (1) they must be aware of each type of decision they should make and (2) they must have an adequate model of each. The second condition, if not the first, is seldom satisfied. The genius of a good manager lies in his ability to manage effectively a system that he does not understand completely. It has long been known in science that the less we understand something, the more variables we require to explain it. Therefore, the manager who is asked what information he needs to control something he does not fully understand usually plays it safe and says he wants as much information as he can get. The MIS designer, who understands the system involved even less than the manager does, adds another safety factor and tries to provide everything. The result is an overload of information, most of which is irrelevant. The greater this overload, the less likely a manager is to extract and use whatever relevant information it contains. The moral is simple: one cannot specify what information is needed for decision making until a valid explanatory model of the decision process and the behavior of the system involved has been constructed.}


\subsection{The history of business intelligence}


\subsection{Challenges with the current solutions}
The current solutions in business intelligence lacks a process support.

The soultions of today lacks a two-way binding to the data, and relies on preaggregated dataware-houses. Occasionally, the analytics boils down to a small selection of objects (e. g. employees, or transactions), but there is no way to navigate to the OLTP part of the application again.

They are integrated in the way they talk together through protocols and nightly batches, but they are not integrated on a top level. They are separate products that needs to be installed and maintained.

Lastly, they don't use metadata at the highest level.

\section{Problem statement, goals, and deliverables}
\label{sec:problem-statement-and-goals}
\begin{secex}
This section should explain what I will to to aid \genus~in solving the above motivation. That is more specifically studying the litterature, and see which techniques are used to enhance OLAP performance, because none of us know too much about what is going on out there. It should state the research question, which should be pretty similar to the Monet thesis.
\end{secex}

For the entire Business Intelligence project, the following goal is defined:
\textbf{Implement Business Intellingence capabilities in \genusSoftware~ that has high performance, handles large datasets, and utilizes the available hardware. The new product must be competitive to it's reference products in terms of performance and fuctionality.}

Working towards this goal does not mean that the newly developed software must be equal in terms of functionality, it is more of a statement that Business Intelligence in \genusSoftware will be compared directly to alternative products, like \qlikview, \tableau, and \powerpivot. In terms of performance and "waiting" it is directly comparable, but lack of functionality in \genusSoftware will be directly be weighted against the benefits of having everything in one solution.

As a first step towards this goal, we address the following research question:
\textbf{RQ1: How to design a high performance database software that is capable of supporting ad-hoc Business Intelligence queries on large datasets?}

The question will be answered by reviewing the current litterature, scrutinizing the competing products, and analysing the bits and bolts needed to develop Business Intelligence capabilities within \genusSoftware. In this thesis, we familiarize ourselvese with the most recent research within the field, and have a look at how reference products solve the problems. 

Based on the results from the RQ1, we continue with the following research question:

\textbf{RQ2: Which of the techiques and concepts are the most promising and pracical to implement in \genusSoftware~in order to reach the goal, and which challenges are identified?}

This question will be answered by combining the results from RQ1 with a brief analysis of the \genusSoftware architecture into a discussion.

%Based on some workshops and discussions with \genus , they have pointed out some key issues they consider important. Based on these key points, I've come up with the following hypotheses:
%\begin{enumerate}\bfseries
%  \item Data needs to be accessed fast. To do this, data has to be stored in a format for quick access. Data access patters need to be analyzed and optimization for data retrieval must be done dynamically.
%  \item The security layer in \genusSoftware~will affect the implementation, and some indexes and formula results will have to be stored per user.
%  \item Indexes, data, and data interchange formats must be compressed to get performance and utilize the system resources efficiently.
%  \item \sout{Formulas must be calculated efficiently to not become a bottleneck.}
%  \item \sout{Network traffic must be reduced by compressing data queries.}
%  \item \sout{In order to create a high-performance system without a database administrator, indexes have to be created dynamically, and there is a question of which and when.}
%\end{enumerate}
%
%Each hypthesis will be studied in the terms of research, competitor solutions, as well as design proposal. In addition to this, I will look for unindentified issues which will affect the design.



