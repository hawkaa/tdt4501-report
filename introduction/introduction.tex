\chapter{Introduction}
\label{chap:introduction}
\clearpage

\section{Background and Motivation}
\label{sec:background-and-motivation}
\begin{secex}
    This section should introduce:
    \begin{enumerate}
      \item \genus
      \item The need for Business Intelligence
      \item The fact that \genus~needs to offer business intelligence to offer a competitive product
      \item Problems with existing products, and they believe \genus~can offer a better product.
    \end{enumerate}
\end{secex}

\genus~is a mid-sized Norwegian company that creates a model-driven application framework that lets you build software applications without programming. The framework lets you define your business object model and configure restrictions and relationsships within the model. Using action orchestration and form design, \genusSoftware can provide a fully functional development framework that can solve a magnitude of problems. It handles millions of rows. \genusSoftware~does also include restrictions on access, which means the data available to the end users in the model is dependent on the user group. \todo{A couple of more sentences on the \genus company and software}

Business Intelligence, or Business Analytics, is an important piece of a business. Business Intelligence allows for making qualified decisions based on data gathered mainly in the IT systems within the company. Traditionally, data warehouses (section X) has been used for this, with preconfigured reports. However, with the advent of in-memory databases and cheap, commodity hardware, several market players has come up with fast, elegant, and end user intuitive solutions to analyze business data. Products like \qlikview, \tableau, and \powerpivot~offers high-performance analytics panel which are easy to configure for the end user, such that important patterns are spotted and acted upon. \todo{Add relevant info about business intelligence and business discovery}

There are several challenges with these products. First of all, they are all separate products, and does not integrate well with existing solutions. This affects the end users, who will have to install and familiarize themselves with a new and unfamiliar user interface. The IT department does also need to specify a separate security system for these applications, such that the data is only available for authorized users. Second, these applications require explicit knowledge about metadata\todo{ask Einar about this}, more specifically; types and database relations needs to be reconfigured when defining the data import routine. Third, in order to handle multiple users at once, these systems require a separate architecture, which is yet another system for the IT staff to install and maintain. In general, \qlikview, \tableau, and \powerpivot~work is isolation and detached to the underlying data.

\genus~wants to offer Business Intelligence capabilities to their customers, because the above challenges cann be overcome by implementing this directly within their \genusSoftware. If implemented correctly, users is able to access Business Intelligence dashboards and data extracts that seamlessly interacts with their main IT system. Security settings and metadata is kept, and the same server architecture can be used.

The motivation for this thesis is to aid \genus~ in the process of adding Business Intelligence capabilities to their software. To do this, we study the relevant litterature in Business Intelligence and OLAP workloads to identify key concepts and challenges in implementing such a system. When such concepts, techniques and challenges are identified, we conclude the paper with which techniques are the most promising and practical in \genus' situation.

\section{Problem statement, goals, and deliverables}
\label{sec:problem-statement-and-goals}
\begin{secex}
This section should explain what I will to to aid \genus~in solving the above motivation. That is more specifically studying the litterature, and see which techniques are used to enhance OLAP performance, because none of us know too much about what is going on out there. It should state the research question, which should be pretty similar to the Monet thesis.
\end{secex}

For the entire Business Intelligence project, the following goal is defined:
\textbf{Implement Business Intellingence capabilities in \genusSoftware~ that has high performance, handles large datasets, and utilizes the available hardware. The new product must be competitive to it's reference products in terms of performance and fuctionality.}

Working towards this goal does not mean that the newly developed software must be equal in terms of functionality, it is more of a statement that Business Intelligence in \genusSoftware will be compared directly to alternative products, like \qlikview, \tableau, and \powerpivot. In terms of performance and "waiting" it is directly comparable, but lack of functionality in \genusSoftware will be directly be weighted against the benefits of having everything in one solution.

As a first step towards this goal, we address the following research question:
\textbf{RQ1: How to design a high performance database software that is capable of supporting ad-hoc Business Intelligence queries on large datasets?}

The question will be answered by reviewing the current litterature, scrutinizing the competing products, and analysing the bits and bolts needed to develop Business Intelligence capabilities within \genusSoftware. In this thesis, we familiarize ourselvese with the most recent research within the field, and have a look at how reference products solve the problems. 

Based on the results from the RQ1, we continue with the following research question:

\textbf{RQ2: Which of the techiques and concepts are the most promising and pracical to implement in \genusSoftware~in order to reach the goal, and which challenges are identified?}

This question will be answered by combining the results from RQ1 with a brief analysis of the \genusSoftware architecture into a discussion.

%Based on some workshops and discussions with \genus , they have pointed out some key issues they consider important. Based on these key points, I've come up with the following hypotheses:
%\begin{enumerate}\bfseries
%  \item Data needs to be accessed fast. To do this, data has to be stored in a format for quick access. Data access patters need to be analyzed and optimization for data retrieval must be done dynamically.
%  \item The security layer in \genusSoftware~will affect the implementation, and some indexes and formula results will have to be stored per user.
%  \item Indexes, data, and data interchange formats must be compressed to get performance and utilize the system resources efficiently.
%  \item \sout{Formulas must be calculated efficiently to not become a bottleneck.}
%  \item \sout{Network traffic must be reduced by compressing data queries.}
%  \item \sout{In order to create a high-performance system without a database administrator, indexes have to be created dynamically, and there is a question of which and when.}
%\end{enumerate}
%
%Each hypthesis will be studied in the terms of research, competitor solutions, as well as design proposal. In addition to this, I will look for unindentified issues which will affect the design.




\subsection{In-Memory}
\label{sub:In-Memory}
We restrict our research to databases that assume the entire dataset can fit in main memory since preliminary research have shown that this is a commonly used approach to achieve good query performance. Systems capable of using main memory as primary storage include \oracle~\cite{Lahiri2015-mz}, \saph~\cite{Farber2012-vh}, \gorilla~\cite{Pelkonen2015-ko}, \qlikview~\cite{Qlik2011-ef}, \tableau~\cite{Kamkolkar2015-iq}, \monetdb~\cite{Boncz2002-yj}, \blink~\cite{Barber2012-xt}, and \sapnw~\cite{Lemke2010-is}. In-memory database systems are where performance and low latency is a key design goal, and on systems that have no need for persistent storage \cite{Zicari2012-is}. Psaroudakis \ea~also say that in-memory databases are easier optimized using parallelization \cite{Psaroudakis2013-fn}. Lastly, a white paper by \qlikview~suggests that companies that are looking for \bi~systems should look for in-memory technologies \cite{Bereanu2010-tj}. 

Several of these systems require the database to fit entirely in memory and does not have a buffer manager. We see in Section \ref{sec:Disk Support} that omiting the buffer manager can increase performance since an extra layer of indirection is removed \cite{Graefe2014-ds}, and it has been shown that databases without buffer managers perform better than those who do \cite{Ferrari2012-hm}. However, databases without buffer managers rely on operating system swapping mechanisms if the database is larger than the provisioned RAM.

According to Kemper \ea, it is safe to assume the entire dataset can fit in memory \cite{Kemper2011-ap} if a large scale server is used. Amazon, one of the biggest commercial enterprises of today, has roughly one billion transactions yearly. If each transaction is stored using 54 bytes, 54 GB is needed to store all transactions for a year, an amount that easily can be accommodated on a single commodity server today. Also, RAM is getting cheaper \cite{Exasol2014-xh}, and together with 64-bits CPUs, in-memory databases are getting an increasingly more significant role \cite{Delaney2014-ip}. Much work has been done in the development of non-volatile RAM, which suggests the era of magnetic disks as primary database storage might soon be over.

\afigure{img/memory-hierarchy.png}{Simplified memory hierarchy. Techniques used to utilize the main memory for a hard-disk based database also applies to utilize CPU caches in a main-memory database. Courtesy of \cite{noauthor_undated-bk}.}{fig:memory-hierarchy}{0.5}
Even though we direct our research to in-memory databases, optimization techniques for disk-based databases will normally apply for in-memory databases as well. As seen in Figure \ref{fig:memory-hierarchy}, techniques used to utilize the main memory for a disk-based database can be applied to utilize CPU caches in a main-memory database. We have only moved a step up in the memory hierarchy \cite{Boncz2002-yj}. There are situations where this argument does not hold and one exception is where optimizations are based on the fact that sequential access is cheaper than random access, which does not apply for RAM.

We continue this research keeping the in-memory assumption true. However, even though RAM is cheap, it is still rarely over-provisioned and unused \cite{Barber2014-ey}. Therefore, we will try to find techniques that keep the memory footprint as low as possible.


\subsection{Read-only}
\label{sub:Read-only}
In the first iteration towards the main goal, we assume a read-only system. By not supporting inserts, updates, and deletes, we simplify the database design. In Section \ref{sec:Write Support and Mixed Workloads}, we explain which techniques that might be applied if write support is required.

We choose to focus on read-only because our primary design goal is performance. A research paper by Psaurodakis \ea~explains how "one size does not fit all" in a database setting, and in order to get good read performance \textit{data freshness}, \textit{query flexibility}, and \textit{query scheduling} must be compromised \cite{Psaroudakis2014-ma}. Figure \ref{fig:compromise}~depicts how performance decreases when write support is added. Also, restricting the research to read-only databases, we limit the scope of the thesis. 

\ffigure{img/compromise.png}{Conceptual figures of how performance of mixed workloads are affected by (a) data freshness, (b) flexibility, and (c) scheduling. Courtesy of \cite{Psaroudakis2014-ma}.}{fig:compromise}

Regarding write support, we distinguish between two types. The first, and most involved, relates to database management systems, which we denote as \textit{direct write support}. \textit{Direct write support} implies that the database directly supports inserts, updates, and deletes. Some systems also support transactions. In such databases, correctness and consistency are the main goals, and data written to the database must immediately be accessible for subsequent queries. In such systems, mutable data structures, like invalidation vectors and delta stores are normally used. We elaborate on these structures in Section \ref{sec:Write Support and Mixed Workloads}.


The other type, which we denote as \textit{update based write support}, is a more light-weight alternative to \textit{direct write support}. Here, inserts, updates, and deletes are periodically merged into the database. In other words, an update done to the database might not be available to succeeding queries immediately. \textit{Update based write support} is used by \qlikview~and \tableau, as they can be configured to subscribe to changes in the database they are connected to.

The advantages of \textit{update based write support} is that it does not rely on mutable data structures like invalidation vectors and delta stores. Instead, immutable data structures optimized for read-only workloads can be used. Since updates in such system happen periodically, instances of read-only structures can be replaced with instances of a more recent snapshot.

We conclude it is safe to focus on optimizations for read-only workloads. Even though a \bd~product must be able to handle updates, a \textit{update based write support} type can be used.

\section{Methodology}
\label{sec:Methodology}
The main part of the literature review was conducted using a method known as \term{Snowballing}, which is convenient if the scope of the project is uncertain \cite{Ang2014-nm}. The Snowballing method is the process whereby you start with a few number of authoritative papers, and based on these you expand your list of readings by relevant work that the papers have cited. The identification of papers can also happen in the other direction, where you look for papers that have used the current one as a reference. Either way, this method is known to generate a large number of papers, so the researcher must be very strict and objective in which papers to read.

The initial papers, theses, and books used in this research were found in collaboration with department staff and regarded in-memory databases, columnar storage, and online analytical processing (OLAP) workloads. Both forward and backward searching was performed, and each paper was considered by reading the abstract, and conclusion and introduction if needed, to be put on the reading list. During the search process, we picked articles that could help us reach \textbf{G1} and answer \textbf{RQ1}. Also, articles must have been published by a known digital library, journal, or conference. 
When the field felt properly understood, we concluded the \term{Snowball} literature study.

The above process yielded a variety of sources relevant to database management systems (DBMS) and OLAP workloads, but we failed to find literature about state-of-the-art commercial database technologies and \bd~products. Hence, in addition to performing a \term{Snowball} literature review, we also scanned the websites of products identified in the previous step. We also looked into products identified in preliminary research by \genus. We searched for whitepapers, blog posts, articles, and forum posts that could reveal how these products answer our research question. Products here include \pn{QlikView}, \pn{Tableau}, \pn{Microsoft SQL Server}, \pn{Oracle Database}, and \pn{EXASOL}.


\section{Deliverables and Contributions}
\label{sec:Deliverables and Contributions}

The main deliverable of this phase of the research is a report summarizing our findings. The report outlines the big picture of OLAP and \bd~product optimizations and studies some interesting cases in detail. The intention of the report is not, and has never been, to be a complete text covering all that is needed to improve \bd~performance. The report is first and foremost written to get an overview of the big picture, inspire, and encourage further studying of referred sources. 

The research has also resulted in many notes, where the report covers roughly one-fourth of these. For convenience, the notes are left out of the report, but they can be used in following parts of the research.

Our contributions of this research are threefold. First, we plan to \textit{improve the evidence} of optimization techniques for OLAP databases. We study techniques category by category, instead of product by product, to get a better a better overview. Besides, this setup enables comparisons between different systems. We know that Abadi \ea~have studied the effects of compression, late materialization, and column store \cite{Abadi2008-dd}. Raman \ea~have studied the impact of SIMD processing in databases \cite{Raman2008-gi}. Sidlauskas \ea~have studied the consequences of how different implementation of the same algorithm affects performance \cite{Sidlauskas2014-ef}. However, this research is limited in scope and fails to provide the big picture for the reader. In our work, we take a more holistic approach.

Second, we plan to \textit{introduce new evidence} to optimizations that are specifically related to \bd~products. Current products do not reveal much about how they work internally. We plan to close the gap between the well studied OLAP databases and commercial \bd~products by relating our findings to in-memory and read-only constraints.

Lastly, our research contributes to an \textit{improved computer-based product}. As explained in Section \ref{sec:Background and Motivation}, we plan to integrate \bd~capabilities within the business' main IT system to overcome the challenges of product isolation, integration, and maintenance. We plan to integrate a system where users can access \bd~dashboards and data extracts that seamlessly interact with their main IT system.

\section{Definitions}
\label{sec:Definitions}

This section presents terms, definitions, and important products that are relevant to understanding the content of this report.

\paragraph{Online Analytical Processing (OLAP)}
\label{par:Online Analytical Processing (OLAP)}
  We use the term Online Analytical Processing (OLAP) extensively in this thesis. By OLAP, we mean systems that enable users to analyze multidimensional data interactively from multiple perspectives \cite{Wikipedia_contributors2015-hw}. OLAP is usually dominated by ad hoc, complex queries that group, aggregate and summarize over large datasets \cite{Bjorklund2011-wh}. Column storage is considered to be an attractive solution for OLAP systems, a technique we study further in Chapter \ref{chap:Data Layout}.


\paragraph{Online Transactional Processing (OLTP)}
\label{par:Online Transactional Processing (OLTP)}
Online Transactional Processing (OLTP) is a class of database systems that manage transaction-oriented applications \cite{Wikipedia_contributors2015-cw}. Transactional workloads are typically referred to as insertion of new records, as well as updates and deletes of single records in the database. An OLTP system normally uses row storage for its data.

\paragraph{Database Management System (DBMS)}
\label{par:Database Management System (DBMS)}
A Database Management System (DBMS) is a computer software application for storage and analysis of data \cite{Wikipedia_contributors2015-pb}. The most common way to interface with a database is through SQL, although other methods exist. Regarding performance, DBMSes can focus on analytical workloads (OLAP), transactional performance (OLTP), or even both. DMBSes do not come with user interfaces for \bd but is designed such that other applications can be built on top of them. In this thesis, we look at \oracle, \ibm, \saph, \sapnw, \mssql, \cstore, \vertica, \blink, \exasol, \oracle, \hyper, and \hyrise.

\paragraph{\bd}
\label{par:Business Discovery}
\bd~is a term introduced by a \qlikview~whitepaper \cite{Qlik2014-vd}. \bd~products differs from traditional \bi~systems by focusing more on the end user. \bd~products does not rely on aggregated data such that the user can follow his "information scent" and click his way through the data. \bd~platforms often provide an architecture that enables panels and dashboards to be shared with multiple clients, both on desktops and mobile devices. Current \bd~products typically build on tailored storage systems that are specifically designed for \bd~workloads, but some of them integrate directly with read-optimized DBMSes. \bd~products include \tableau, \qlikview, \powerpivot, and more. \bd~is explained in greater detail is Section~\ref{sec:Business Discovery}.

\paragraph{Reference products}
\label{par:Reference products}
We will occasionally use the term \textit{reference products} in this report. By reference products we mean \bd~products pointed out by \genus~that \bd~capabilities in \genusSoftware~will be compared to. In this report, we study \qlikview~and \tableau.

\paragraph{\exasol}
\label{par:exasol}
\exasol~is important in this report because as of November 2015, it is the highest performing DBMS in the official TPC-H benchmark \cite{noauthor_undated-vr}. This benchmark test database performance for analytical workloads on a predefined set of queries. \exasol outperforms other systems tested in this benchmark by a factor of 10 in average. We, therefore, study techniques used by this system in extra detail.


\section{Report Outline}
\label{sec:Thesis Outline}
We structure our report as following:
\begin{itemize}
  \item \textbf{Chapter 2} introduces background material relevant to this report. Here, we study \bi, \bd, and testing.
  \item \textbf{Chapter 3-6} go into details techniques that are paramount to OLAP and \bd~performance, namely \textit{data layout}, \textit{data compression}, \textit{indexes and auxiliary structures}, and \textit{parallelization}.
  \item \textbf{Chapter 7} studies techniques relevant for query processing in \bd~applications in terms of \textit{joining}, \textit{grouping and aggregation}, and \textit{query optimization}.
  \item \textbf{Chapter 8} shows that careful implementation is required for a system to perform well.
  \item \textbf{Chapter 9} elaborates on other considerations in a \bd~application, like \textit{application design}, \textit{denormalization}, \textit{disk support}, and \textit{write support}.
  \item \textbf{Chapter 10} concludes this report and points at interesting directions for future work.
\end{itemize}

