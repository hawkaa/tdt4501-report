\chapter{Introduction}
\label{chap:introduction}
\clearpage

\section{Background and Motivation}
\label{sec:background-and-motivation}
\begin{secex}
    This section should introduce:
    \begin{enumerate}
      \item \genus
      \item The need for Business Intelligence
      \item The fact that \genus~needs to offer business intelligence to offer a competitive product
      \item Problems with existing products, and they believe \genus~can offer a better product.
    \end{enumerate}
\end{secex}

\genus~is a mid-sized Norwegian company that creates a model-driven application framework that lets you build software applications without programming. The framework lets you define your business object model and configure restrictions and relationsships within the model. Using action orchestration and form design, \genusSoftware can provide a fully functional development framework that can solve a magnitude of problems. It handles millions of rows. \genusSoftware~does also include restrictions on access, which means the data available to the end users in the model is dependent on the user group. \todo{A couple of more sentences on the \genus company and software}

Business Intelligence, or Business Analytics, is an important piece of a business. Business Intelligence allows for making qualified decisions based on data gathered mainly in the IT systems within the company. Traditionally, data warehouses (section X) has been used for this, with preconfigured reports. However, with the advent of in-memory databases and cheap, commodity hardware, several market players has come up with fast, elegant, and end user intuitive solutions to analyze business data. Products like \qlikview, \tableau, and \powerpivot~offers high-performance analytics panel which are easy to configure for the end user, such that important patterns are spotted and acted upon. \todo{Add relevant info about business intelligence and business discovery}

There are several challenges with these products. First of all, they are all separate products, and does not integrate well with existing solutions. This affects the end users, who will have to install and familiarize themselves with a new and unfamiliar user interface. The IT department does also need to specify a separate security system for these applications, such that the data is only available for authorized users. Second, these applications require explicit knowledge about metadata\todo{ask Einar about this}, more specifically; types and database relations needs to be reconfigured when defining the data import routine. Third, in order to handle multiple users at once, these systems require a separate architecture, which is yet another system for the IT staff to install and maintain. In general, \qlikview, \tableau, and \powerpivot~work is isolation and detached to the underlying data.

\genus~wants to offer Business Intelligence capabilities to their customers, because the above challenges cann be overcome by implementing this directly within their \genusSoftware. If implemented correctly, users is able to access Business Intelligence dashboards and data extracts that seamlessly interacts with their main IT system. Security settings and metadata is kept, and the same server architecture can be used.

The motivation for this thesis is to aid \genus~ in the process of adding Business Intelligence capabilities to their software. To do this, we study the relevant litterature in Business Intelligence and OLAP workloads to identify key concepts and challenges in implementing such a system. When such concepts, techniques and challenges are identified, we conclude the paper with which techniques are the most promising and practical in \genus' situation.

\section{Problem statement, goals, and deliverables}
\label{sec:problem-statement-and-goals}
\begin{secex}
This section should explain what I will to to aid \genus~in solving the above motivation. That is more specifically studying the litterature, and see which techniques are used to enhance OLAP performance, because none of us know too much about what is going on out there. It should state the research question, which should be pretty similar to the Monet thesis.
\end{secex}

For the entire Business Intelligence project, the following goal is defined:
\textbf{Implement Business Intellingence capabilities in \genusSoftware~ that has high performance, handles large datasets, and utilizes the available hardware. The new product must be competitive to it's reference products in terms of performance and fuctionality.}

Working towards this goal does not mean that the newly developed software must be equal in terms of functionality, it is more of a statement that Business Intelligence in \genusSoftware will be compared directly to alternative products, like \qlikview, \tableau, and \powerpivot. In terms of performance and "waiting" it is directly comparable, but lack of functionality in \genusSoftware will be directly be weighted against the benefits of having everything in one solution.

As a first step towards this goal, we address the following research question:
\textbf{RQ1: How to design a high performance database software that is capable of supporting ad-hoc Business Intelligence queries on large datasets?}

The question will be answered by reviewing the current litterature, scrutinizing the competing products, and analysing the bits and bolts needed to develop Business Intelligence capabilities within \genusSoftware. In this thesis, we familiarize ourselvese with the most recent research within the field, and have a look at how reference products solve the problems. 

Based on the results from the RQ1, we continue with the following research question:

\textbf{RQ2: Which of the techiques and concepts are the most promising and pracical to implement in \genusSoftware~in order to reach the goal, and which challenges are identified?}

This question will be answered by combining the results from RQ1 with a brief analysis of the \genusSoftware architecture into a discussion.

%Based on some workshops and discussions with \genus , they have pointed out some key issues they consider important. Based on these key points, I've come up with the following hypotheses:
%\begin{enumerate}\bfseries
%  \item Data needs to be accessed fast. To do this, data has to be stored in a format for quick access. Data access patters need to be analyzed and optimization for data retrieval must be done dynamically.
%  \item The security layer in \genusSoftware~will affect the implementation, and some indexes and formula results will have to be stored per user.
%  \item Indexes, data, and data interchange formats must be compressed to get performance and utilize the system resources efficiently.
%  \item \sout{Formulas must be calculated efficiently to not become a bottleneck.}
%  \item \sout{Network traffic must be reduced by compressing data queries.}
%  \item \sout{In order to create a high-performance system without a database administrator, indexes have to be created dynamically, and there is a question of which and when.}
%\end{enumerate}
%
%Each hypthesis will be studied in the terms of research, competitor solutions, as well as design proposal. In addition to this, I will look for unindentified issues which will affect the design.




\subsection{In-Memory}
\label{sub:In-Memory}
We restrict our research to databases that assume the entire dataset can fit in main memory, since preliminary research have shown that this is a commonly used approach to achieve good query performance. Systems capable of using main memory as primary storage include \oracle~\cite{Lahiri2015-mz}, \saph~\cite{Farber2012-vh}, \gorilla~\cite{Pelkonen2015-ko}, \qlikview~\cite{Qlik2011-ef}, \tableau~\cite{Kamkolkar2015-iq}, \monetdb~\cite{Boncz2002-yj}, \blink~\cite{Barber2012-xt}, and \sapnw~\cite{Lemke2010-is}. In-memory database systems can be used where performance and low latency is a key design goal, and on systems that has no need for persistent storage \cite{Zicari2012-is}. It has also been said that in-memory databases are easier optimized using parallelization \cite{Psaroudakis2013-fn}. Lastly, a white paper by \qlikview~suggests that companies looking for \bi~systems should look for in-memory technologies \cite{Bereanu2010-tj}. 

Several of these systems require the database to fit entirely in memory, and does not have a buffer manager. We see in Section \ref{sec:Disk Support} that omiting the buffer manager can increase performance since an extra layer of indirection is removed \cite{Graefe2014-ds}, and it has been shown that databases without buffer managers performs better than those who do \cite{Ferrari2012-hm}. However, databases without buffer managers rely on operating system swapping mechanisms if the database is larger than the provisioned RAM.

According Kemper \ea, it is safe to assume the entire dataset can fit in memory \cite{Kemper2011-ap} if a large scale server is used. Amazon, one of the largest commercial enterprises, has $~1$ billion yearly transactions, and if each transaction is stored using 54 bytes, 54 GB is needed to store all transactions for a year. In addition to this, RAM is getting cheaper \cite{Exasol2014-xh}, and together with 64-bits CPUs \cite{Delaney2014-ip}, in-memory databases are making an increasingly more important role. Much progress has been done in the development of non-volatile RAM, which suggests the era of magnetic disks as the primary database storage hardware might soon be over. \todo{Rephrase paragraph}

It is worth noting that even though we focus on in-memory performance, optimizations performed for disk-based databases will still be applicable. We have just moved one step up through the memery hierarchy \cite{Boncz2002-yj}. That is, optimizations done to better utilize available memory for a disk based database will help an in-memory database utilize available CPU caches. We are aware that certain optimizations are better suited for disk-based databases than for in-memory solutions, however since most research does not specify if they emphasize on disk or in-memory performance, we believe their optimizations contribute to both databaes types. \todo{Make this much clearer}

We continue this research keeping the in-memory assumption true. However, even though DRAM is cheap, it is still rarely over-provisioned and unused \cite{Barber2014-ey}. Therefore, we find techniques that try to keep the memory footprint as low as possible.

\subsection{Read-mostly and Read-only}
\label{sub:Read-mostly and Read-only}
In the first iteration towards the main goal, we assume that the system will be read-only. By not supporting inserts, updates, and deletes, we simplify the database design. In Section \ref{sec:Write-support and mixed workloads}, we explain which techniques might be applied if we need write support.

\ffigure{img/compromise.png}{Conceptual figures of how performance of mixed workloads are affected by (a) data freshness, (b) flexibility, and (c) scheduling. Courtesy of \cite{Psaroudakis2014-ma}.}{fig:compromise}
We choose to focus on read-only because our main design goal is performance. A research paper by Psaurodakis \ea~explains how "one size does not fit all" in a database setting, and in order to get good read performance, data freshness, query flexibility, and query scheduling must be compromised \cite{Psaroudakis2014-ma}. Figure \ref{fig:compromise}~depicts how performance decreases when write support is added. In addition, restricting the research to read-only databases, we limit the scope of the thesis. 

However, we relax the requirement that updates in the underlying data source must be available in our \bd~application once they are performed.

Write support can be separated into two types. The first type is the most involved and first and foremost relates to DBMSes. This type supports inserts, updates, and deletes directly, and in some cases transactions. In these systems, correctness and consistency is the key goal.

The other type which is a more light-weight alternative to the above, is to have the \bd~solution listen to updates performed by another system, say a DBMS. These solutions are not guaranteed to be timely, but timely enough for most cases. \qlikview~and \tableau~supports the latter. If write support is to be considered for \bd~in \genusSoftware, this type is considered first.

To conclude this section, for all practical reasons, we may focus on read-only techniques and can several places assume immutable structures. We do however need some sort of updates, but this can happen in batches or through periodical merge, but we do not need updates in the data to be available to our application immediately. \todo{Adjust entire section to read-mostly.}

\section{Methodology}
\label{sec:Methodology}
The main part of the literature review was conducted using a method known as \term{Snowballing}, which is convenient if the scope of the project is uncertain \cite{Ang2014-nm}. The Snowballing method is the process whereby you start with a few number of authoritative papers, and based on these you expand your list of readings by relevant work that the papers have cited. The identification of papers can also happen in the other direction, where you look for papers that have used the current one as a reference. Either way, this method is known to generate a large number of papers, so the researcher must be very strict and objective in which papers to read.

The initial papers, theses, and books used in this research were found in collaboration with department staff and regarded in-memory databases, columnar storage, and online analytical processing (OLAP) workloads. Both forward and backward searching was performed, and each paper was considered by reading the abstract, and conclusion and introduction if needed, to be put on the reading list. During the search process, we picked articles that could help us reach \textbf{G1} and answer \textbf{RQ1}. Also, articles must have been published by a known digital library, journal, or conference. 
When the field felt properly understood, we concluded the \term{Snowball} literature study.

The above process yielded a variety of sources relevant to database management systems (DBMS) and OLAP workloads, but we failed to find literature about state-of-the-art commercial database technologies and \bd~products. Hence, in addition to performing a \term{Snowball} literature review, we also scanned the websites of products identified in the previous step. We also looked into products identified in preliminary research by \genus. We searched for whitepapers, blog posts, articles, and forum posts that could reveal how these products answer our research question. Products here include \pn{QlikView}, \pn{Tableau}, \pn{Microsoft SQL Server}, \pn{Oracle Database}, and \pn{EXASOL}.


\section{Contributions}
\label{sec:Contributions}
\todo{Add from research plan}
The novelty of this thesis is two-fold. Most related work presesnts a single system and explains the techniques use there. Here, we present the litterature by category, not by product. This way, comparisons can be made more easily, and simplifies the process to getting an overview over where the litterature agrees, and where there are contradictions. We have found work X and work Y. These systems present single concepts, but neither of these span as broadly as we do in this thesis.

The second contribution, is a thorough discussion on where the various techniques presented applies. \genusSoftware have several fundamental and practical limitations, and based on these, We present the techniques that are the most fruitful. Optimizations in columnar storage have already been investigated by Abadi \ea~\cite{Abadi2008-dd}, however this work focus on the fundamental differences between row and column stores. We present a more holistic picture, considering more concepts and techniques. In addition to this, this thesis specifically targets \genusSoftware.

\section{Definitions}
\label{sec:Definitions}
This section presents terms, definitions, and important products that are relevant to understading the content of this thesis.

\paragraph{Online Analytical Processing (OLAP)}
\label{par:Online Analytical Processing (OLAP)}
Online Analytical Processing (OLAP) is a term thet we use extensively in this thesis. OLAP is defined as systems capable of analytical queries, typically aggregations over large ranges of rows and groping in one or multiple dimensions. OLAP workloads are typically read-only queries.\todo{Source and improve}

\paragraph{Online Transactional Processing (OLTP)}
\label{par:Online Transactional Processing (OLTP)}
Online Transactional Processing (OLTP) is often considered as the opposite of OLAP. OLTP is defined as systems capable of transactional queries, typically inserts, updates, and deletes. \todo{Source and improve}

\paragraph{Database Management System (DBMS)}
\label{par:Database Management System (DBMS)}
We look into several Database Management Systems (DBMS) in this thesis. These systems are general purpose systems for storage of data. DBMSes can focus on read performance for analytical workloads, write performance for transactional workloads, or both. These systems do not come with user interfaces for \bd, but is designed such that other applications can be built on top of them. The normal interface with the DBMSes is SQL. In this thesis, we look at \oracle, \ibm, \saph, \sapnw, \mssql, \cstore, \vertica, \blink, \exasol, \oracle, \hyper, and \hyrise.

\paragraph{\bd}
\label{par:Business Discovery}
\bd~is a term introduced by \qlikview~\cite{Qlik2014-vd}. \bd~is different from traditional \bi~by focusing more on the end user. By not relying on preaggregated data, the user can follow his own "information scent" and click his way through the data. \bd~platforms delivers panels and dashboards to multiple devices, and allows for easy sharing. They typically build on storage systems that are specifically designed for \bd~workloads, but some of them integrate directly with read-optimized DBMSes. \bd~products include \tableau, \qlikview, \powerpivot, and more. \bd~is explained in greater detail is Section~\ref{sec:Business Discovery}.

\paragraph{Reference products}
\label{par:Reference products}
By reference products we mean product pointed out \genus, products which \bd~in \genusSoftware~will be directly compared to. Studied in this thesis are \qlikview~and \tableau.


\paragraph{\exasol}
\label{par:exasol}
Although not a definition per se, \exasol~is important in this thesis because it is as of November 2015 the highest performing database for the official TPC-H benchmark \cite{noauthor_undated-vr}. This benchmark test datas database performance for analytical workloads on a predefined set of queries on various database sizes ranging from 100 GB to 100,000 GB. \exasol outperforms other systems tested in this benchmark by a factor of 10 in average.

\section{Report Outline}
\label{sec:Thesis Outline}
We structure our report as following:
\begin{itemize}
  \item \textbf{Chapter 2} introduces background material relevant to this report. Here, we study \bi, \bd, and testing.
  \item \textbf{Chapter 3-6} go into details techniques that are paramount to OLAP and \bd~performance, namely \textit{data layout}, \textit{data compression}, \textit{indexes and auxiliary structures}, and \textit{parallelization}.
  \item \textbf{Chapter 7} studies techniques relevant for query processing in \bd~applications in terms of \textit{joining}, \textit{grouping and aggregation}, and \textit{query optimization}.
  \item \textbf{Chapter 8} shows that careful implementation is required for a system to perform well.
  \item \textbf{Chapter 9} elaborates on other considerations in a \bd~application, like \textit{application design}, \textit{denormalization}, \textit{disk support}, and \textit{write support}.
  \item \textbf{Chapter 10} concludes this report and points at interesting directions for future work.
\end{itemize}



