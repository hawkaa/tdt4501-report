\section{Background on Business Intelligence Software}
\label{sec:bi-background}
\textit{For managers to know what information they need (1) they must be aware of each type of decision they should make and (2) they must have an adequate model of each. The second condition, if not the first, is seldom satisfied. The genius of a good manager lies in his ability to manage effectively a system that he does not understand completely. It has long been known in science that the less we understand something, the more variables we require to explain it. Therefore, the manager who is asked what information he needs to control something he does not fully understand usually plays it safe and says he wants as much information as he can get. The MIS designer, who understands the system involved even less than the manager does, adds another safety factor and tries to provide everything. The result is an overload of information, most of which is irrelevant. The greater this overload, the less likely a manager is to extract and use whatever relevant information it contains. The moral is simple: one cannot specify what information is needed for decision making until a valid explanatory model of the decision process and the behavior of the system involved has been constructed.}


\subsection{The history of business intelligence}


\subsection{Challenges with the current solutions}
The current solutions in business intelligence lacks a process support.

The soultions of today lacks a two-way binding to the data, and relies on preaggregated dataware-houses. Occasionally, the analytics boils down to a small selection of objects (e. g. employees, or transactions), but there is no way to navigate to the OLTP part of the application again.

They are integrated in the way they talk together through protocols and nightly batches, but they are not integrated on a top level. They are separate products that needs to be installed and maintained.

Lastly, they don't use metadata at the highest level.
