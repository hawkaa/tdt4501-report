\chapter{Indexes and Auxiliary Structures}
\label{chap:Indexes and Auxiliary Structures}
Although the number indexes and materialized views can greatly be reduced by using compressed column storage \cite{Lahiri2015-mz}, \bd~products may still benefit from indexes and other data structures that aid query processing. In this chapter, we look at indexes used for single value lookups, index structures that increase predicate evaluation performance, and join indexes. We also briefly look into caching and preaggregation of results.

Indexes are extensively used in \qlikview. According to a whitepaper on scalability, RAM normally need 2\%-10\% of the application on disk to accomodate overhead such as indexes and data associations.

\newpage

\section{Value Lookups}
\label{sec:Value Lookups}
\ffigure{img/dictionary.png}{Example of a dictionary compression implementation with inverted indexes. Courtesy of \cite{Psaroudakis2015-lc}.}{fig:dictionary}
\afigure{img/static-inverted-index.png}{A static, dictionary column indexed using an inverted indexes structure. The dictionary (D) is is used to look up positions (P) in the column (AV) using an offset array (O). Courtesy of \cite{Schwalb2014-hn}.}{fig:static-inverted-index}{0.4}
Even though most OLAP queries cover a large number of rows, there are situations where a position, or positions, for a single value need to be looked up. This situation benefits from an index structure. For such situations, inverted indexes are normally used \cite{Lemke2010-is}. An implementation of inverted indexes for a dictionary encoded column is shown in Figure \ref{fig:dictionary}. 

\hyrise~exploits the fact that the columns are immutable, such that the inverted indexes can be optimized in terms of storage size and cache awareness \cite{Schwalb2014-hn}. This index leverages the read-only assumption by creating an immutable structure, as seen in Figure \ref{fig:static-inverted-index}. This implementations  maps dictionary entries to an offset vector $O$, which is used to look up value positions in vector $P$. For composite keys on multiple columns, a separate dictionary used only for the index is used to map entries to the offset vector. When using this structure, the dictionary keys for the columns included in the index must be looked up prior to checking the composite index dictionary.

The inverted index structure in \hyrise~is only is suited for situations where the columns are immutable, as adding elements to this will require the entire structure to be rebuilt \cite{Schwalb2014-hn}. Hence, for columns supporting inserts, deletes and updates, tree-based indexes are used instead. \mssql~does not use inverted indexes, only hash and BW-tree indexes \cite{Delaney2014-ip, noauthor_undated-vq}.

However, the cost of keeping inverted indexes must always be weighted against the benefits \cite{Lemke2010-is}, since inverted indexes normally take more space than the documents they are storing \cite{Moffat1992-tz}. If memory footprint must be kept low, and none of the queries benefit from value lookups, systems will benefit from dropping inverted indexes.

\subsection{Index Lookup vs Table Scan}
\label{sub:Index Lookup vs Table Scan}
Although an index structure exist in the database, there are several situations where a table scan is better than looking up multiple values using an index. Boncz \ea~\cite{Boncz2006-md} shows how easily a table scan should be triggered, and that index based lookups actually can degrade performance. Several researches indicate this \cite{Boncz2002-yj, Abadi2008-dd}. The work of Holloway \ea~\cite{Holloway2008-rr} sees an increased importance of a full table scan. \qlikview~has also been reported to calculate much of the results via scans \cite{noauthor_undated-js}. \todo{Back this section up with more theory}

Early versions of \blink~had a design goal to avoid structures that benefit particular queries. Hence, they skipped indexes and calculated all the results on denormalized tables via scans.

\section{Bitmap indexes}
\label{sec:Bitmap indexes}
A Bitmap index is a special structure where each distinct value at a column is represented as a bitmap where all rows containing that value has the value 1. Bitmap indexes has two main purposes: Column compression, and easy lookup and combination of predicates. Work from 1999 \cite{Witten1999-qq} claims that compressed inverted indexes are almost always superior to bitmap indexes in practical situations. However, Bjørklund claims that bitmap indexes are widely used in Decision Support Systems \cite{Bjorklund2011-wh}. 

Bitmap indexes has been used in \term{Information Retrieval} software, but is it now replaced by inverted indexes \cite{Bjorklund2011-wh}. They are faster and generally outperform bitmap indexes. For single value lookups, inverted indexes is normally used instead of bitmap indexes \cite{Moffat1992-tz}.

Bitmap indexes is suited for OLAP \cite{Stonebraker2005-qz}

\genusSoftware~reference product \qlikview~use binary indexes for each fields, which is the same as bitmap indexes \cite{Qlik2011-ef}. \qlikview~also utilizes some sort of bitmap indexes, which are created for each field \cite{Qlik2011-ef}.

\subsection{Compression of bitmaps}
\label{sub:Compression of bitmaps}
\ffigure{img/wah.png}{Example WAH compression. Courtesy of \cite{Bjorklund2011-wh}.}{fig:wah}
Bitmap indexes are another way of compressing the data, especially on unsorted columns with few distinct values \cite{Stonebraker2005-qz}. Since the bitmaps normally are sparse, they can be compressed with techniques like WAH \cite{Bjorklund2011-wh} (Figure \ref{fig:wah}) or Run-Length Encoding. However, inverted indexes are normally always instead of bitmaps \cite{Witten1999-qq}

Another way of compressing bitmaps is using a parameterized ways, which is explored by \cite{Moffat1992-tz}. Although this article is old, it still indicates that the total number of 1's in a bitmaps should be the main parameter when considering which compression method to use.

Hierarchical compression of bitmaps might be used as well \cite{Witten1999-qq}. This method is recommended, since it is very fast to see if a row is present in the bitmap. The key is either way to have fast extraction of single values.
The first, is that it can work as column compression. \cstore~uses bitmaps to compress columns if they have few distinct values. Since the bitmaps normally are sparse, they can be compressed even further by applying run-length encoding.


Another way of compressing a bitmap index is to use WAH compression \cite{Bjorklund2011-wh}, which is a form of run-length encoding. Columns with high cardinality are well represented in this scheme. An implementation of WAH is \algmet{FastBit}.

\subsection{Bloom Filters}
\label{sub:Bloom Filters}
Bitmaps can be made smaller at the expense of false matches \cite{Witten1999-qq}. This can be used for an inner loop join \todo{Add from Bratsbergs book}.

\section{Table Statistics}
\label{sec:Table Statistics}
Table statistics can be used to skip certain parts of the data based on attributes, like minimum and maximum value. This is especially handy when the columns are horizontally partitioned, like mentioned in Section \ref{sub:Horizontal partitioning of columns}. Partitions can be skipped on the basis of column maximum or mininum value, or the per-block dictionary can be scanned to see if a certain key is present.

One way to keep track of this metadata is through a synopsis table. \ibm uses this, and the table keeps control over pages with corresponding metadata such that pages can be skipped.

The storing of block metadata for quick data pruning is used by \oracle~\cite{Lahiri2015-mz}, \ibm~\cite{Roman2013-em}, \vertica~\cite{Lamb2012-kg}, \monetx~\cite{Boncz2005-wj}, \mssql~\cite{Larson2013-mc}, and \exasol~\cite{Exasol2014-xh}.

To calculate table statistics, histograms of the table value distributions can be made. This is done by \ibm~\cite{Raman2013-em, Raman2008-gi} and \mssql~\cite{Larson2013-mc}. To make these histograms, random sampling can be used, and here two techniques apply. The first one is truly random, where values are picked across the whole column. The second one is a grouped version, where a random sample range is picked.

\section{Join indexes}
\label{sec:Join indexes}
\afigure{img/join-index.png}{Example join index structure in \cstore. Courtesy of \cite{Stonebraker2005-qz}.}{fig:join-index}{0.4}
Join indexes is a structure that stores precomputed join results, and are discussed in several articles. Join indexes are used by \monetdb~\cite{Boncz2002-yj} and \monetx~\cite{Boncz2005-wj}. \todo{Add information about the impact in memory using this technique.} 

\cstore uses join indexes as it allows for quick joining \cite{Stonebraker2005-qz}, but in \vertica, the commercialized version of \cstore, the cost of join indexes was outweighed by the benefits \cite{Lamb2012-kg}.

For join indexes, it is common to use bitmap indexes \cite{Bjorklund2011-wh}. We discuss bitmap indexes in Section \ref{sec:Bitmap Indexes}.

\section{Data duplication and preaggregation}
\label{sec:Data duplication and preaggregation}
Another way to boost performance, is through data duplication. \cstore~and \vertica~allows for storing the columns in multiple sort orders at the same time \cite{Stonebraker2005-qz, Lamb2012-kg}. \blink allows for storing columns in multiple banks, such that predicates can be evaluated in the same bank \cite{Johnson2008-cp}. \exasol~replicates tables across nodes in the system if they are small enough.

Earlier, data was preaggregated to achieve good performance, but this has been considered as inflexible and one of the main challenges with old ROLAP and MOLAP systems \cite{Boncz2002-yj}. Modern systems does not preaggregate results, like \sapnw~\cite{Lemke2010-is} and \qlikview~\cite{Qlik2014-vd}, and calculate all results as needed. \qlikview does however allow data to be preaggregated when loaded into memory \cite{Qlik2011-yc}.

\subsection{Column projections} MOVE TO INDEX, CACHING ETC
\label{sub:Column projections} 
C-store allows to have several projections per column, with different sort order. They claim the redundant storage is justified by the additional compression. Configuring these projections requires a DBA. \pn{Vertica}, a commercialized version of \pn{C-Store} supports several narrow projections in addition to one super projection.

\section{Dynamic generation of indexes}
\label{sec:Dynamic generation of indexes}
Indexes are used extensively by \exasol, where they are produces, maintained, and deleted as needed \cite{Exasol2014-xh}. This gives faster filter operations.

Boncz \ea~\cite{Boncz2006-md} also mentions "index on-the-fly" briefly.

\section{Caching}
\label{sec:Caching}
In order to improve performance, caching might be used. In dedicated database systems, like the \exasol, joins, queries, and aggregates are cached to keep the system scalable \cite{Exasol2014-xh, Plattner2014-fr}. The cache can be put in the delta store, mentioned in Section \ref{sub:Delta Store}.

For read-only, business intelligence engines, caching can be used more aggressively. In \qlikview, the server has a central cache function such that calculations only need to be done once \cite{Qlik2011-ef}. This gives better user experience and lower the CPU footprint. When deciding which results to cache, the system takes into consideration how long it takes to generate the results \cite{noauthor_undateds-js}. In addition, cache is shared among users \cite{Qlik2011-yc}. Hence, if adding "one user at a time", the cache builds up, and the system can gradually serve more and more users. In \tableau, data and calculations are cached and shared among the users \cite{Kamkolkar2015-iq}. \vertipaq~also cache data, and this improves performance compared to \mssql~\cite{Ferrari2012-hm}.

\section{Partition Indexing}
\label{sec:Partition Indexing}
If columns are horizontally partitioned, as discussed in Section \ref{sec:Horizontal Partitioning}, an index structure might be required to look up the partitions. \ibm~uses a B+ tree for this purpose \cite{Raman2013-em}.

