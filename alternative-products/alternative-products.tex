\chapter{Alternative Products}
\label{chap:alternative-products}

This chapter will contain info about:

\begin{itemize}
  \item QlikView
  \item Oracle In-Memory option
  \item Tableu
\end{itemize}

\section{Microsoft, IBM, SAP, and Oracle}
Microsoft, SAP, IBM, and Oracle does all supply OLAP solutions. There has been several blog posts and articles (\textit{insert references here}) on which of these products are the best. Lots of these blog posts are very biased, and there is little doubt they're written from a marketers perspective. However, there's been little doubt that they all offer high performance analytics solutions. What is common for all of them is:
\begin{itemize}
  \item Columnar storage
  \item In-memory storage
  \item Indexing
\end{itemize}

They all seem to agree on the above points.

However, some people claim that these solutions are not pure In-Memory Database Systems (IMDS) (cite Steve Graves). They are only 100\%~cached databases tweaked to exploit the available memory.

\section{Microsoft SQL Server}
Starting from SQL Server 2012, Microsoft offers column storage \cite{noauthor_undated-py}, known as \textit{in-memory columnstore index}. From SQL Server 2016, the index is now updateable and available for tables that are stored in-memory. Previous versions were read-only \cite{noauthor_undated-ut}. They clain their product achieve up to 10x query performance and 10x data compression compared to a row-oriented storage. Microsoft claim the columnstore indexes give high performance on analytic queries that scan large amounts of data.

\paragraph{Row Groups}
\label{par:Row Groups}
Microsoft SQL server operate on \textit{row groups} which are groups of rows compressed into a columnar format. The number of rows in a row group must me small enough to benefit from in-memory operations and large enough to achive high compression rates.

\paragraph{Batch execution}
\label{par:Batch execution}
This is a method for processing multiple rows in a query together. Typically, the query performance improves by 2-4x. 

\paragraph{Dual format}
\label{par:Dual format}
Like Oracle, a row store and a column store can be combined, such that the transactional workload can use the row store and analytics run on the column store. Since the column store is updated when data changes in the row store, both indexes are working against the same data.

\paragraph{Transactional support}
\label{par:Transactional support}
Snapshot isolation, online defragmentation of indexes. Filters for column store data, such that only cold data can be accessed. \cite{noauthor_undated-tc}

\paragraph{Indexing types}
\label{par:Indexing types}
SQL server supports two kind of indexes \cite{Delaney2014-ip}:
\begin{itemize}
  \item Non-clustered hash-indexes, stored as hash tables with linked lists
  \item nonclustedred indexes. Supports retrieving ranges of values, ordering of rows and performance for queries using the inequality operation.
\end{itemize}

\paragraph{Native compiled code}
\label{par:Native compiled code}
SQL server compiles code into DLL and runs it to operate on the data directly. \todo{Fill in more here}

\paragraph{deltastore}
\label{par:deltastore}
SQL server keeps a data structure called the \textit{deltastore} to manage changes in the data. This structure gathers data up to a certain threshold. When the threshold is reached, the deltastore closes and the column store index is rebuilt.



\section{Oracle In-Memory option}
\label{sec:Oracle In-Memory option}

\subsection{Key design concepts}
\label{sub:Key design concepts}
This section will enumerate the key design concepts in the in-memory option for the oracle database

\paragraph{Datadase object selection}
\label{par:Datadase object selection}
Oracle offers a wide range of in-memory granularities for their users. Their \texttt{INMEMORY} attribute can be specified on any oracle database object, like a tablespace, table, partition or materialized view. Being able to select partitions for in-memory usage, allows you to maintain the newest data in-memory, while rows older than a certain date will be stored on disk. For a table or tablespace, certain columns can be excluded in the data loaded into memory.

\todo{Check out what a database object is}

\paragraph{Column store}
\label{par:Column store}
Oracle saves the data in the dual format, that is both row and columns of data. The analytic queries uses the column format.

\paragraph{Background worker population}
\label{par:Background worker population}
The oracle database uses a background worker to populate the IM storage, and will be able to handle queries even if the data is not loaded into memory yet. The background workers columize and compress the data continously.

\paragraph{In-Memory Compression units}
\label{par:In-Memory Compression units}
The column store is made up of multiple extents, called In-Memory Compression Units (IMCUs). Data is loaded into these units without sorting, they are put in the same way as it appears in the row format.

\paragraph{Priorities}
\label{par:Priorities}
All database objects that can be loaded into memory can have a priority associated with it (five levels). When selecting which objects to load into memory, these priorities are used. All objects at a given priority level must be fully populated before the population for any objects at a lower priority level can commence.

\paragraph{In-Memory Compression}
\label{par:In-Memory Compression}
Oracle utilizes a compression format on their in-memory storage layout (\cite{Oracle2015-fs}). Compression is normally considered as a space-saving mechanism, but compression will also improve query performance. All scanning and filtering operations operate directly on the compressed columns which decreases query execution time. Data is decompressed before it is required for the result set.

Oracle offers several different compression levels.\todo{Is this needed?}

We have been unable to find the specific compression algorithm, but Oracle says that Dictionary Encoding, Run Length Encoding and Bit-Packing is used \cite{Oracle2015-fs}.

\paragraph{Indexes}
\label{par:Indexes}
Indexes for each of the columns in the column store are maintained by storing the minimum and maxumum value for each compression unit (IMCU). If the query specifies a \texttt{WHERE} clause, all ICMUs indexes are checked. If the column value is outside the maximum and the minimum value for an ICMU, the IMCU will not be scanned.

Additionally, when a dictionary based compression is used, the dictionary can be used to determine if the value being searched for actually exists within an IMCU.

\paragraph{SIMD Vector Processing}
\label{par:SIMD Vector Processing}
SIMD (Single Instruction, Multiple Data) instructions are used within the Oracle database to process multiple coulmn values at the same time using a single CPU instruction. 

\paragraph{In-Memory joins}
\label{par:In-Memory joins}
Oracle uses Bloom Filters to join two or more tables.

\paragraph{In-Memory aggregation}
\label{par:In-Memory aggregation}
To perform summaries and aggregations, Oracle applies an optimizer transformation, called \textit{Vector Group By}
\todo{Provide an explanation of the vector group by here}

\subsection{Other}
\label{sub:Other}
Oracle offers high performance OLTP with transaction support. There are background workers to keep as few stale tuples in the IMCUs.

Oracle scales out in a RAC environment. Each node will have parts of the data, and is considered as a share-nothing architecture.

Fault tolerance. Improves performance and increases availability.

All of the information in this section was found in paper 1, paper 2, paper 3.

\section{Other}
\label{sec:Other}
Other techiques not tied to a specific software package has also been developed

\subsection{Late Materialization}
\label{sub:Late Materialization}
Abadi et. al. \cite{Abadi2008-dd} explains how column storages benefit from late materialization, which means tuples are not constructed before they are really needed. The advantages of this technique is four-fold. \todo{Add information about the paper here}

\subsection{Invisible Join}
\label{sub:Invisible Join}
\todo{Explain the invisible join from the Abadi et al. paper}
\section{Conclusions}
\label{sec:Conclusions}
Since \genusSoftware supports several different database backends, and among them providers like MySQL which does not support in-memory database tables, one can not utilize the DBMS technology which exists. \genus must develop their own system, preferably a pure IMDS.
